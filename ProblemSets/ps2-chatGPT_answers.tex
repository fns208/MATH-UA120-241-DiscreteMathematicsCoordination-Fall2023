\documentclass{article}
% This is a LaTeX file.  It is a text file that is compiled
% by a program called LaTeX into a pretty PDF file.  
% If you're viewing this file on Overleaf, you'll see that PDF 
% in the window to the right.
%
% The LaTeX macro language is complicated, so we have inserted
% lots of documenting comments into the file.  Comments start
% with `%' and continue to the end of the line.  In Overleaf's
% window, they are colored green.
%
% Comments prefixed with `Student:' are relevant to students.
% Skip anything else you don't understand, or ask me.
%
% set font encoding for PDFLaTeX or XeLaTeX
\usepackage{ifxetex}
\ifxetex
  \usepackage{fontspec}
\else
  \usepackage[T1]{fontenc}
  \usepackage[utf8]{inputenc}
  \usepackage{lmodern}
\fi

% Student: These lines describe some document metadata.
\title{Problem Set 2}
\author{%
% Student: change the next line to your name!
    Name
\\  MATH-UA 120 Discrete Mathematics
}
\date{due September 22, 2023}


\usepackage[headings=runin-fixed-nr]{exsheets}
% These make enumerates within questions start at the second ("(a)") level, rather than the first ("1.") level.
\makeatletter
    \newcommand{\stepenumdepth}{\advance\@enumdepth\@ne}
\makeatother
\SetupExSheets{
    question/pre-body-hook=\stepenumdepth,
    solution/pre-body-hook=\stepenumdepth,
}
\DeclareInstance{exsheets-heading}{runin-nn-np}{default}{
    runin = true,
    title-post-code = .\space,
    join = {
        main[r,vc]title[l,vc](0pt,0pt);
    }
}
\newif\ifshowsolutions
% Student: replace `false' with `true' to typeset your solutions.
% Otherwise they are ignored!
\showsolutionstrue
\ifshowsolutions
    \SetupExSheets{
        question/pre-hook=\itshape,
        solution/headings=runin-nn-np,
        solution/print=true,
        solution/name=Answer
    }%
    \makeatletter%
    \pretocmd{\@title}{Answers to }%
    \makeatother%
\else
    \SetupExSheets{solution/print=false}
\fi

% Bug workaround: http://tex.stackexchange.com/a/146536/1402
%\newenvironment{exercise}{}{}
\RenewQuSolPair{question}{solution}
%\let\answer\solution
%\let\endanswer\endsolution
\usepackage{manfnt}
\newcommand{\danger}{\marginpar[\hfill\dbend]{\dbend\hfill}}

\usepackage{amsmath, amsthm}
\usepackage{amsfonts}
\usepackage{enumerate}
\usepackage{siunitx}
\DeclareSIUnit\pound{lb}
\usepackage{hyperref}
\newtheorem*{theorem}{Theorem}
\newtheorem*{claim}{Claim}
\theoremstyle{definition}
\newtheorem*{definition}{Definition}
% We are creating a command "\xor".
\newcommand{\xor}{\underline{\lor}}
% This is the beginning of the part of the file that describes
% the text of the document.
% That's why it says `\begin{document}' below. :-)
\begin{document}
\maketitle



These are to be written up and turned in to Gradescope.\\



\ifshowsolutions
    \SetupExSheets{solution/print=true}
\else
    \danger
 \underline{ \LaTeX  Instructions:}  You can view the source (\texttt{.tex}) file to get some more examples of \LaTeX{} code.  I have commented the source file in places where new \LaTeX{} constructions are used.
  
  Remember to change \verb|\showsolutionsfalse| to \verb|\showsolutionstrue|
    in the document's preamble 
    (between \verb|\documentclass{article}| and \verb|\begin{document}|)
\fi

\section*{Assigned Problems}


\begin{question}
    \begin{enumerate}
        \item Disprove: Every triangle has at least one obtuse angle.
        \item Disprove: For all real numbers $x$, $2^x\geq x+1$.
        \item Disprove: For every positive non-prime integers $n$, if some prime $p$ divides $n$, 
            then some other prime $q$ ($q\neq p$) also divides $n$.
        \item Disprove: For all integers $n$, if $n^5-n$ is even, then $n$ is even.
    \end{enumerate}
\end{question}
% Student: put your answer between the next two lines.
\begin{solution}
    \begin{enumerate}
        \item Disproof for the first statement: Consider an equilateral triangle where all angles are $60^\circ$, making them all acute.
        \item Disproof for the second statement: Take $x = -1$, then $2^x = \frac{1}{2}$, but $x + 1 = 0$.
        \item Disproof for the third statement: Take $n = 4$; it's a positive non-prime integer, divisible by the prime $2$, but no other prime divides it.
        \item Disproof for the fourth statement: Take $n = 1$; then $n^5 - n = 0$ which is even, but $n = 1$ is not even.
    \end{enumerate}
\end{solution}


\begin{question}
   (Scheinerman, Exercise 7.12:)
    Another method to prove that certain Boolean formulas
    are tautologies is to use the properties in Theorem~7.2
    together with the fact that $x \rightarrow y$ is equivalent
    to $(\neg x) \vee y$ (Proposition~7.3)
    For example, Exercise 7.11, part (b) asks you to establish that the formula $(x \wedge (x \rightarrow y)) \rightarrow y$ is 
    a tautology.  Here is a derivation of that fact:
    % This is an "align" environment for aligning
    % mathematical equations.  Ampersands (&) denote 
    % separation between columns.  The first one means
    % there will be alignment around the equal sign.
    % The double-ampersands put a second column, left-
    % justified.  Double-backslashes (\\) separate lines.
    \begin{align*}
        (x \wedge (x \rightarrow y)) \rightarrow y
        &= [x \wedge (\neg x \vee y)] \rightarrow y
        && \text{translate $\rightarrow$} 
        \\
        &= [(x \wedge \neg x) \vee (x \wedge y)] \rightarrow y
        && \text{distributive}
        \\
        &= [\mathrm{FALSE} \vee (x\wedge y)] \rightarrow y
        && \text{inverse elements}
        \\
        &= (x\wedge y) \rightarrow y
        && \text{identity element}
        \\
        &= \neg(x\wedge y) \vee y
        && \text{translate $\rightarrow$}
        \\
        &= (\neg x \vee \neg y) \vee y
        && \text{De~Morgan's laws}
        \\
        &= \neg x \vee (\neg y \vee y)
        && \text{associativity}
        \\
        &= \neg x \vee \mathrm{TRUE}
        && \text{inverse elements}
        \\
        &= \mathrm{TRUE}
        && \text{identity element}
        \\
    \end{align*}
    Use this technique [not truth tables] to prove that these formulas
    are tautologies:
    \begin{enumerate}
        \item $(x \rightarrow \mathrm{FALSE}) \rightarrow \neg x$
        \item $(x \rightarrow y) \wedge (x \rightarrow \neg y) \rightarrow \neg x$
    \end{enumerate}
\end{question}
% Student: put your answer between the next two lines.
\begin{solution}
    \begin{enumerate}
        \item Using the equivalence $x \rightarrow y \equiv (\neg x) \vee y$, we can rewrite the given formula:
        \[
        (x \rightarrow \mathrm{FALSE}) \rightarrow \neg x \equiv ((\neg x) \vee \mathrm{FALSE}) \rightarrow \neg x \equiv x \rightarrow \neg x.
        \]
        Now, using the property $x \rightarrow \neg x$ is always a tautology.
        
        \item Using the equivalence $x \rightarrow y \equiv (\neg x) \vee y$, we can rewrite the given formula:
        \begin{align*}
        (x \rightarrow y) \wedge (x \rightarrow \neg y) \rightarrow \neg x &\equiv ((\neg x) \vee y) \wedge ((\neg x) \vee (\neg y)) \rightarrow \neg x\\
        &\equiv \neg x \vee (y \wedge (\neg y)) \rightarrow \neg x\\
        &\equiv \neg x \vee \mathrm{FALSE} \rightarrow \neg x \\
        &\equiv \neg x \rightarrow \neg x.
        \end{align*}
        Now, using the property $\neg x \rightarrow \neg x$ is always a tautology.
    \end{enumerate}
\end{solution}



\begin{question}
	Let the following statements be given.
		\begin{align*}
		p &= \text{``Jeremiah is hungry.''}\\
		q &= \text{``The refrigerator is empty.''}\\
		r &= \text{``Jeremiah is mad.''}
		\end{align*}
	\begin{enumerate}
		\item Rewrite the following statement as a Boolean expression.\\
			\begin{center}
			If Jeremiah is hungry and the refrigerator is empty, then Jeremiah is mad.
			\end{center}
		\item Construct a truth table for the statement in part (a).
		\item Suppose that the statement given in part (a) is true, and suppose also that Jeremiah is not mad and the refrigerator is empty. Is Jeremiah hungry? Justify your answer using the truth table.
	\end{enumerate}
\end{question}
% Student: put your answer between the next two lines.
\begin{solution}
    \begin{enumerate}
        \item Boolean expression for the given statement: $(p \land q) \rightarrow r$.
        \item Truth table for the statement in part (a):
        \[
        \begin{array}{cccc}
            p & q & r & (p \land q) \rightarrow r \\
            \hline
            T & T & T & T \\
            T & T & F & F \\
            T & F & T & T \\
            T & F & F & F \\
            F & T & T & T \\
            F & T & F & F \\
            F & F & T & T \\
            F & F & F & T \\
        \end{array}
        \]
        \item Based on the truth table, when $p = F$, $q = T$, and $r = F$, the statement is true. Therefore, if Jeremiah is not mad and the refrigerator is empty, he is not hungry.
    \end{enumerate}
\end{solution}

% Student: The following is an example of how to typeset a table
	% This is an "array" environment for aligning
	% individual objects. It is similar to "align"
	% Right after we \begin{array}, we indicate
	% how we want to align each individual column:
	% "c" for centered, "r" for right-justified,
	% "l" ("ell") for left-justified.
	% The total number of c's, r's, and l's should match
	% the number of columns.
	% We can use this to form a table. To create the borders,
	% we use "|" (the notation for "divides") to create a
	% vertical line between each column.
	% After each line, double-backslashes (\\) separate lines.
	% If we want a horizontal line after each line, 
	% we use "\hline".
	%For math environment, we can use "$$" or "\[ \]".
	%\[\begin{array}{| c | c || c |}
	%\hline
	%x & y & x \vee y \\
	%\hline
	%T & T & T \\ 
	%T & F & T \\
	%F & T & T \\
	%F & F & F \\
	%\hline
	%\end{array}\]

\begin{question}
    Suppose each single character stored in a computer uses eight bits. 
    Then each character is represented by a different sequence of eight 0's and 1's called a bit pattern, 
    such as 10011101 or 00100010. Provide a brief reasoning for the following questions.
        \begin{enumerate}
            \item How many different bit patterns are there?
            \item How many different bit patterns are palindromes (the same backwards as forwards)?
            \item How many different bit patterns have an even number of 1's?
            \item How many different bit patterns have the property that their second and fourth digits are 1's?
            \item How many different bit patterns have the property that their second or fourth digits are 1's?
        \end{enumerate}
\end{question}
% Student: put your answer between the next two lines.
\begin{solution}
    \begin{enumerate}
        \item There are $2^8 = 256$ different bit patterns since each bit can be either 0 or 1, giving 2 choices, and there are 8 bits.
        \item To form a palindrome, the first 4 bits can be any pattern, and the last 4 bits will be the reverse of the first 4. So, there are $2^4 = 16$ palindromes.
        \item To have an even number of 1's, we can choose either 0, 2, 4, 6, or 8 ones. There are $\binom{8}{0} + \binom{8}{2} + \binom{8}{4} + \binom{8}{6} + \binom{8}{8} = 1 + 28 + 70 + 28 + 1 = 128$ such patterns.
        \item The second and fourth digits being 1's means that the pattern is in the form \texttt{\_\_1\_\_1\_\_}. For the two empty positions, there are $2^2 = 4$ choices (0 or 1).
        \item The second or fourth digit being 1's means that the pattern is in the form \texttt{\_\_1\_\_1\_\_}. For the two empty positions, there are $2^2 = 4$ choices (00, 01, 10, or 11).
    \end{enumerate}
\end{solution}


\begin{question}
    Suppose we want to make a list of length 5 from the letters $A, B, C, D, E, F, G, H, I, J$. Provide a brief reasoning for the following questions.
        \begin{enumerate}
            \item How many such lists can be made if repetition is not allowed and the list must begin with a vowel?
            \item How many such lists can be made if repetition is not allowed and the list must end with a vowel?
            \item How many such lists can be made if repetition is not allowed and the list must contain exactly one $A$?
            \item How many such lists can be made if repetition is not allowed and the list must contain exactly two vowels?
        \end{enumerate}
\end{question}
% Student: put your answer between the next two lines.
\begin{solution}
    \begin{enumerate}
        \item To form a list of length 5 where repetition is not allowed and it begins with a vowel, there are $2 \times 9 \times 8 \times 7 \times 6$ possibilities. We have 2 choices for the first vowel (A or E), and then we have 9 choices for the remaining positions (any of the 9 letters excluding the chosen vowel), 8 choices for the next, and so on.
        \item To form a list of length 5 where repetition is not allowed and it ends with a vowel, there are $9 \times 8 \times 7 \times 6 \times 2$ possibilities. We have 9 choices for the first position (excluding the last vowel), and then we have 8 choices for the next, and so on. Finally, we have 2 choices for the last position (A or E).
        \item To form a list of length 5 where repetition is not allowed and it contains exactly one A, there are ${5 \choose 1} \times 9 \times 8 \times 7 \times 6$ possibilities. We first choose the position for A (any of the 5 positions), then we have 9 choices for the first remaining position, 8 choices for the next, and so on.
        \item To form a list of length 5 where repetition is not allowed and it contains exactly two vowels, there are ${5 \choose 2} \times 5 \times 4 \times 7 \times 6$ possibilities. We first choose 2 positions for the vowels (out of 5), then we have 5 choices for the first vowel, 4 choices for the second, and 7 choices for the remaining consonants.
    \end{enumerate}
\end{solution}


\end{document}
