\documentclass{article}
% The LaTeX macro language is complicated, so we have inserted
% lots of documenting comments into the file.  Comments start
% with `%' and continue to the end of the line.  In Overleaf's
% window, they are colored green
%
% Comments prefixed with `Student:' are relevant to students.
% Skip anything else you don't understand, or ask me.
%
% set font encoding for PDFLaTeX or XeLaTeX
\usepackage{ifxetex}
\ifxetex
  \usepackage{fontspec}
\else
  \usepackage[T1]{fontenc}
  \usepackage[utf8]{inputenc}
  \usepackage{lmodern}
\fi

% Student: These lines describe some document metadata.
\title{Problem Set 7}
\author{%
% Student: change the next line to your name!
    Name
\\  MATH-UA 120 Discrete Mathematics
}
\date{due December 1, 2023}


\usepackage[headings=runin-fixed-nr]{exsheets}
% These make enumerates within questions start at the second ("(a)") level, rather than the first ("1.") level.
\makeatletter
    \newcommand{\stepenumdepth}{\advance\@enumdepth\@ne}
\makeatother
\SetupExSheets{
    question/pre-body-hook=\stepenumdepth,
    solution/pre-body-hook=\stepenumdepth,
}
\DeclareInstance{exsheets-heading}{runin-nn-np}{default}{
    runin = true,
    title-post-code = .\space,
    join = {
        main[r,vc]title[l,vc](0pt,0pt);
    }
}
\newif\ifshowsolutions
% Student: replace `false' with `true' to typeset your solutions.
% Otherwise they are ignored!
\showsolutionstrue
\ifshowsolutions
    \SetupExSheets{
        question/pre-hook=\itshape,
        solution/headings=runin-nn-np,
        solution/print=true,
        solution/name=Answer
    }%
    \makeatletter%
    \pretocmd{\@title}{Answers to }%
    \makeatother%
\else
    \SetupExSheets{solution/print=false}
\fi

% Bug workaround: http://tex.stackexchange.com/a/146536/1402
%\newenvironment{exercise}{}{}
\RenewQuSolPair{question}{solution}
%\let\answer\solution
%\let\endanswer\endsolution
\usepackage{manfnt}
\newcommand{\danger}{\marginpar[\hfill\dbend]{\dbend\hfill}}

% We are creating a command for some common commands.
\newcommand{\Z}{\mathbb{Z}}
\newcommand{\N}{\mathbb{N}}
\newcommand{\im}{\operatorname{im}}
\newcommand{\id}{\operatorname{id}}

% This package is for specifying graphics.  It's amazing.
% Manual at http://texdoc.net/texmf-dist/doc/generic/pgf/pgfmanual.pdf
\usepackage{tikz}

\usepackage{amsmath, amsthm}
\usepackage{amsfonts}
\usepackage{siunitx}
\DeclareSIUnit\pound{lb}
\usepackage{hyperref}
\newtheorem*{theorem}{Theorem}
\theoremstyle{definition}
\newtheorem*{definition}{Definition}
% This is the beginning of the part of the file that describes
% the text of the document.
% That's why it says `\begin{document}' below. :-)
\begin{document}
\maketitle



These are to be written up in \LaTeX{} and turned in to Gradescope.\\



\ifshowsolutions
    \SetupExSheets{solution/print=true}
\else
    \danger
 \underline{ \LaTeX{}  Instructions:}  You can view the source (\texttt{.tex}) file to get some more examples of \LaTeX{} code.  I have commented the source file in places where new \LaTeX{} constructions are used.
  
  Remember to change \verb|\showsolutionsfalse| to \verb|\showsolutionstrue|
    in the document's preamble 
    (between \verb|\documentclass{article}| and \verb|\begin{document}|)
\fi

\section*{Assigned Problems}



\begin{question}
        A fair coin is flipped 10 times.
    \begin{enumerate}
        \item What is the probability that there are an equal number of head and
            tails?
        \item What is the probability the first three flips are heads?
        \item What is the probability that there are an equal number of heads
            and tails and the first three flips are heads?
        \item What is the probability that there are an equal number of heads
            and tails or the first three flips are heads (or both)?
        \item What is the probability that the first three flips are heads given
            that an equal number of heads and tails are flipped?
    \end{enumerate}
\end{question}
% Student: put your answer between the next two lines.
\begin{solution}
    \begin{enumerate}
        \item $\frac{\binom{10}{5}}{2^{10}} = \frac{252}{1024}$.  There are
            $2^{10}$ outcomes when a fair coin is flipped 10 times.  The number
            of heads and tails are equal only if 5 heads and 5 tails are
            flipped.  There are $\binom{10}{5}$ ways to flip exactly 5 heads.
        \item $\frac{2^7}{2^{10}} = \frac{1}{8}$, since the outcomes of the
            remaining 7 flips is $2^7$.
        \item $\frac{\binom{7}{2}}{2^{10}} = \frac{21}{1024}$, since exactly 2
            heads must be flipped from the remaining 7 flips is denoted by
        $\binom{7}{2}$.
        \item Let $A$ be the event exactly 5 heads are tossed.  Let $B$ be the
            event the first three flips are heads.  By inclusion-exclusion
            principle, $P(A\cup B) = P(A) + P(B) - P(A\cap B)= \frac{1}{8} +
            \frac{252}{1024} - \frac{21}{1024} = \frac{359}{1024}$.
        \item By the definition of conditional probability,
        \[
            P(A \mid B) = \frac{P(A \cap B)}{P(B)} = \frac{\frac{21}{1024}}{\frac{252}{1024}} =
                \frac{21}{252}.
        \]
    \end{enumerate}
   
   {\color{red} Rubric:
\begin{itemize}
\item 2P per part
\end{itemize}}
\end{solution}

\begin{question}
    An unfair coin shows heads with probability $p$ and tails with probability
    $1-p$. Suppose this coin is flipped $2$ times. Let $A$ be the event that the coin comes up first heads and
    then tails.  Let $B$ be the event that the coin comes up first tails and
    then heads.
    \begin{enumerate}
        \item  Find $P(A)$.
        \item  Find $P(B)$.
        \item  Find $P(A \mid A \cup B)$.
        \item  Find $P(B \mid A \cup B)$.
    \end{enumerate}
    Explain how one could use this unfair coin to make a fair decision.
\end{question}
% Student: put your answer between the next two lines.
\begin{solution}
    \begin{enumerate}
        \item $p(1-p)$, since the second flip is independent from the first.
        \item $(1-p)p$, since the second flip is independent from the first.
        \item Observe that $A \cap B = \emptyset$.  By addition principle, $P(A
        \cup B) = P(A )+ P(B) = 2p(1-p)$. By the definition of conditional
        probability,
            \[
                P(A \mid A \cup B) = \frac{P(A \cap (A \cup B))}{P(A\cup B)}
                = \frac{P(A)}{P(A\cup B)} = \frac{p(1-p)}{2p(1-p)} = \frac{1}{2}.
            \]
        \item Similarly,
            \[
                P(B  \mid  A \cup B) = \frac{P(B \cap (A \cup B))}{P(A\cup B)}
                    = \frac{P(B)}{P(A\cup B)} = \frac{p(1-p)}{2p(1-p)} = \frac{1}{2}.
            \]
    \end{enumerate}
    We can make a fair decision from an unfair coin as follows:
    \begin{itemize}
        \item Flip the coin twice.
        \item If the results match, start over and flip again.
        \item If the results differ, use the first result, forgetting the second.
    \end{itemize}
    Based on the computations above, we know that given that the results differ
    (i.e. $A\cup B$), the probability of flipping heads (i.e. event $A$) or
    tails (i.e. event $B$) at the first toss is the same, making it a ``fair"
    toss.
    
{\color{red} Rubric:
\begin{itemize}
\item part a: 1P
\item part b: 1P
\item part c: 3P
\item part d: 3P
\item fair coin explanation: 2P
\end{itemize}}
\end{solution}

\begin{question}
    Suppose that $A$ and $B$ are events in a sample space $(S,P)$.
    Prove or disprove:
    \begin{enumerate}
        \item If $P(A \cap B)=0$, then $P(A\mid B)=P(B\mid A)$ if and only
            if $P(A)=P(B)$.
        \item If $P(A)>0, P(B)>0$ but $P(A \cap B)=0$, then $P(A\mid
            B)=P(B\mid A)$.  If proven, give an example of two such events
            with $P(A) \ne P(B)$.
    \end{enumerate}
\end{question}
% Student: put your answer between the next two lines.
\begin{solution}
 \textit{Counterexample for (a):}
    Roll a fair six-sided die and record the numbed faced-up. Let $A$ be the
    event of rolling $1$ or $2$ and let $B$ be the event of rolling $4$, $5$
    or $6$.  Observe that $A \cap B = \emptyset$. By the definition of
    conditional probability, $P(A \mid B) = P(B \mid A) = 0$. On the other
    hand, $P(A) = \frac{1}{3}$ and $P(B) = \frac{1}{2}$. Since $P(A \mid B)
    = P(B \mid A) = 0$, but $\frac{1}{3} = P(A) \ne P(B) = \frac{1}{2}$, the
    statement is false.
    
\begin{proof}[Proof of (b)]
    Suppose that $A$ and $B$ are arbitrary events in a sample space $(S,P)$.
    Suppose that $P(A)>0, P(B)>0$ and $P(A \cap B)=0$.  By the definition of
    conditional probability,
    \[
        P(A \mid B)P(B) = P(A \cap B) = 0.
    \]
    Since $P(B) >0$, it must necessarily hold $P(A\mid B) = 0$.  Similarly,
    we have that
    \[
        P(B\mid A)P(A) = P(A \cap B) = 0
    \]
    and since $P(A) >0$, it must be that $P(B\mid A) = 0$. It follows that
    $P(A \mid B) = 0 = P(B \mid A)$.
\end{proof}
{\color{red} Rubric:
\begin{itemize}
\item 5P per part
\end{itemize}}
\end{solution}

\begin{question}
    Consider the sample space $S = \{ a, b, c \}$, with equal probability for each outcome. Define the random variables $X$ and  
    $Y$ by $X(a) = -1, \quad X(b) = 0, \quad X(c) = 1, \quad Y(a) = Y(c) = 0, \quad Y(b) = 1.$ 
    Check that $\mathrm{Var}(X+Y) = \mathrm{Var}(X) + \mathrm{Var}(Y)$, but that $X$ and $Y$ are not independent.
\end{question}
% Student: put your answer between the next two lines.
\begin{solution}
	We first have
	\[
	E(X = -1) = E(X = 0) = E(X = 1) = 1/3, \quad E(Y = 0) = \frac23, \quad E(Y = 1) = 1/3.
	\]
	We deduce
	\[
	E(X) = \frac13 \left ( -1 + 0 + 1 \right ) = 0, \quad E(Y) = \frac23 \times 0 + \frac13 \times 1 = \frac13,
	\]
	and
	\[
	E(X^2) = \frac13 \left ( (-1)^2 + 0^2 + 1^2 \right ) = \frac23, \quad E(Y^2) = \frac23 \times 0^2 + \frac13 \times 1^2 = \frac13,
	\]
	so $\mathrm{Var}(X) = 2/3$ and $\mathrm{Var}(Y) = 2/9$. 
	
	On the other hand, we have $(X+Y)(a) = -1$, $(X+Y)(b) = 1$, and $(X+Y)(c) = 1$, so
	\[
	E(X+Y = -1) = \frac13, \quad E(X+Y = 1) = \frac23.
	\]
	We deduce
	\[
	E(X+Y) = \frac13 \times (-1) + \frac23 \times 1 = \frac13
	\]
	(or we could use linearity), and
	\[
	E((X+Y)^2) = \frac13 \times (-1)^2 + \frac23 \times 1^2 = 1,
	\]
	so $\mathrm{Var}(X+Y) = 8/9$, which is indeed $\mathrm{Var}(X) + \mathrm{Var}(Y)$.
	
	Finally, $X$ and $Y$ are not independent since
	\[
	E(X = 0, Y = 0)  = 0 
	\]
	which is not equal to
	\[
	E(X = 0) E(Y = 0) = \frac13 \times \frac23 = \frac 29.
	\]
	
{\color{red} Rubric:
\begin{itemize}
\item Variance Computation: 8P
\item Independence Explanation: 2P
\end{itemize}}
\end{solution}

\begin{question}
    Provide an alternative proof to Proposition 31.7 using any of the statements in Proposition 31.8.
    \begin{theorem}[Proposition 31.7]
    Let $A$ and $B$ be events in a sample space $(S, P)$. Then 
    \[ P(A) + P(B) = P(A\cup B) + P(A\cap B).\]
    \end{theorem}
\end{question}
% Student: put your answer between the next two lines.
\begin{solution}
Let $A$ and $B$ be events in a sample space $(S, P)$. Note 
\begin{equation}
P(A\cup B) = P(A\cup ( \bar{A} \cap B)) = P(A) + P(\bar{A} \cap B),
\end{equation}
since $A$ and $\bar{A}\cap B$ are pairwise disjoint. Since $B = ( A\cap B) \cup (\bar{A} \cap B)$ and $A\cap B$ and $ \bar{A} \cap B$ are pairwise disjoint, we have
\begin{equation}
P(B) = P((A\cap B) \cup (\bar{A} \cap B)) = P(A\cap B) +P(\bar{A} \cap B).
\end{equation}

Meaning, $P(\bar{A} \cap B) = P(B)-P(A\cap B)$. Substituting equation (2) into equation (1) gives us
\[
P(A\cup B) = P(A) + P(B)-P(A\cap B),
\]

which is equivalent to  $P(A) + P(B) = P(A\cup B) + P(A\cap B)$.

{\color{red} Rubric:
\begin{itemize}
\item Follow RVF rubric with 1P for \LaTeX
\end{itemize}}
\end{solution}

\end{document}
