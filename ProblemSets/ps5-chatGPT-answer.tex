\documentclass{article}
% The LaTeX macro language is complicated, so we have inserted
% lots of documenting comments into the file.  Comments start
% with `%' and continue to the end of the line.  In Overleaf's
% window, they are colored green
%
% Comments prefixed with `Student:' are relevant to students.
% Skip anything else you don't understand, or ask me.
%
% set font encoding for PDFLaTeX or XeLaTeX
\usepackage{ifxetex}
\ifxetex
  \usepackage{fontspec}
\else
  \usepackage[T1]{fontenc}
  \usepackage[utf8]{inputenc}
  \usepackage{lmodern}
\fi

% Student: These lines describe some document metadata.
\title{Problem Set 5}
\author{%
% Student: change the next line to your name!
    Name
\\  MATH-UA 120 Discrete Mathematics
}
\date{due October 27, 2023}


\usepackage[headings=runin-fixed-nr]{exsheets}
% These make enumerates within questions start at the second ("(a)") level, rather than the first ("1.") level.
\makeatletter
    \newcommand{\stepenumdepth}{\advance\@enumdepth\@ne}
\makeatother
\SetupExSheets{
    question/pre-body-hook=\stepenumdepth,
    solution/pre-body-hook=\stepenumdepth,
}
\DeclareInstance{exsheets-heading}{runin-nn-np}{default}{
    runin = true,
    title-post-code = .\space,
    join = {
        main[r,vc]title[l,vc](0pt,0pt);
    }
}
\newif\ifshowsolutions
% Student: replace `false' with `true' to typeset your solutions.
% Otherwise they are ignored!
\showsolutionstrue
\ifshowsolutions
    \SetupExSheets{
        question/pre-hook=\itshape,
        solution/headings=runin-nn-np,
        solution/print=true,
        solution/name=Answer
    }%
    \makeatletter%
    \pretocmd{\@title}{Answers to }%
    \makeatother%
\else
    \SetupExSheets{solution/print=false}
\fi

% Bug workaround: http://tex.stackexchange.com/a/146536/1402
%\newenvironment{exercise}{}{}
\RenewQuSolPair{question}{solution}
%\let\answer\solution
%\let\endanswer\endsolution
\usepackage{manfnt}
\newcommand{\danger}{\marginpar[\hfill\dbend]{\dbend\hfill}}

% We are creating a command for some common commands.
\newcommand{\Z}{\mathbb{Z}}

% This package is for specifying graphics.  It's amazing.
% Manual at http://texdoc.net/texmf-dist/doc/generic/pgf/pgfmanual.pdf
\usepackage{tikz}

\usepackage{amsmath, amsthm, amssymb}
\usepackage{amsfonts}
\usepackage{siunitx}
\DeclareSIUnit\pound{lb}
\usepackage{hyperref}
\newtheorem*{theorem}{Theorem}
\theoremstyle{definition}
\newtheorem*{definition}{Definition}
% This is the beginning of the part of the file that describes
% the text of the document.
% That's why it says `\begin{document}' below. :-)
\begin{document}
\maketitle



These are to be written up in \LaTeX{} and turned in to Gradescope.\\



\ifshowsolutions
    \SetupExSheets{solution/print=true}
\else
    \danger
 \underline{ \LaTeX{}  Instructions:}  You can view the source (\texttt{.tex}) file to get some more examples of \LaTeX{} code.  I have commented the source file in places where new \LaTeX{} constructions are used.
  
  Remember to change \verb|\showsolutionsfalse| to \verb|\showsolutionstrue|
    in the document's preamble 
    (between \verb|\documentclass{article}| and \verb|\begin{document}|)
\fi

\section*{Assigned Problems}

\begin{question}
    Prove the following statement by contrapositive: \\
    For all $n\in \mathbb{N}$, if $2^n<n!$, then $n>3$.
\end{question}
% Student: put your answer between the next two lines.
\begin{solution}
We will prove the statement by contrapositive. The contrapositive of "If $2^n < n!$, then $n > 3$" is "If $n \leq 3$, then $2^n \geq n!$."

We will show this for $n \leq 3$:

\begin{description}
\item[Case 1: $n = 1$] 
For $n = 1$, $2^n = 2$ and $n! = 1$. Since $2 \geq 1$, the statement holds for $n = 1$.

\item[Case 2: $n = 2$]
For $n = 2$, $2^n = 4$ and $n! = 2$. Since $4 \geq 2$, the statement holds for $n = 2$.

\item[Case 3: $n = 3$]
For $n = 3$, $2^n = 8$ and $n! = 6$. Since $8 \geq 6$, the statement holds for $n = 3$.

Now we have shown that for all $n \leq 3$, the statement "If $n \leq 3$, then $2^n \geq n!$" is true. Therefore, by proving the contrapositive, we have shown that the original statement "If $2^n < n!$, then $n > 3$" is also true.
\end{description}
Hence, the statement is proven by contrapositive.
\end{solution}



\begin{question}
    Prove the following by contradiction:\\
    Let $A, B, C$ be sets. If $A\subseteq B$ and $B\cap C=\emptyset$, then $A\cap C=\emptyset$.
\end{question}
% Student: put your answer between the next two lines.
\begin{solution}
We will prove the statement by contradiction. Let $A, B, C$ be sets such that $A\subseteq B$ and $B\cap C = \emptyset$. We assume the negation of the statement, which is:

$\neg(A\cap C = \emptyset)$

This means there exists an element $x$ such that $x\in A$ and $x\in C$. In other words, $x\in A\cap C$. But this contradicts our assumption that $A\cap C = \emptyset$, so the negation is false.

Since the negation of the statement is false, the original statement is true. Therefore, we have proven that if $A\subseteq B$ and $B\cap C=\emptyset$, then $A\cap C=\emptyset$.
\end{solution}


\begin{question}
    Prove the following statement by contradiction:\\
    Let $x, y\in \Z$. Then $x^2-4y-3\neq 0$.
\end{question}
% Student: put your answer between the next two lines.
\begin{solution}
We will prove the statement by contradiction. Let $x, y$ be integers, and assume, for the sake of contradiction, that $x^2 - 4y - 3 = 0$. 

Now, we will consider this equation and its properties. First, we can rewrite the equation as follows:

$x^2 = 4y + 3$

This implies that $x^2$ is an odd integer because it is equal to $4y + 3$, where $4y$ is an even integer and adding $3$ to it makes it an odd integer. 

We know that for any integer $x$, $x^2$ is always non-negative. Thus, for $x^2$ to be an odd integer, it must be greater than or equal to $1$. 

However, this is a contradiction, as our assumption was that $x^2 - 4y - 3 = 0$. 

Therefore, our initial assumption that $x^2 - 4y - 3 = 0$ must be false. In other words, $x^2 - 4y - 3 \neq 0$ for all integers $x$ and $y$.

Hence, the statement has been proven by contradiction.
\end{solution}


\begin{question}
    Prove the following by smallest counterexample:\\
    Let $n\in \mathbb{N}$. If $n\geq 1$, then $4 \mid (5^n-1)$.
\end{question}
% Student: put your answer between the next two lines.
\begin{solution}
We will prove the statement by the method of smallest counterexample, also known as proof by infinite descent.

Assume, for the sake of contradiction, that there exists a natural number $k$ such that $4$ does not divide $(5^k - 1)$, and $k$ is the smallest such number. In other words, $5^k - 1$ is not divisible by $4$, and for all natural numbers $m$ such that $1 \leq m < k$, $5^m - 1$ is divisible by $4$.

Now, let's consider the value of $5^k$. Since $5^k$ is not divisible by $4$ (because $5^k - 1$ is not divisible by $4$), it must leave a remainder of $1$ or $3$ when divided by $4$. In other words, $5^k \equiv 1 \pmod{4}$ or $5^k \equiv 3 \pmod{4}$.

Case 1: $5^k \equiv 1 \pmod{4}$

In this case, we have $5^k \equiv 1 \pmod{4}$, which implies that $5^k - 1 \equiv 0 \pmod{4}$. However, this contradicts our assumption that $5^k - 1$ is not divisible by $4$. Therefore, this case is not possible.

Case 2: $5^k \equiv 3 \pmod{4}$

In this case, we have $5^k \equiv 3 \pmod{4}$, which implies that $5^k - 1 \equiv 2 \pmod{4}$. This means $5^k - 1$ is divisible by $2$ but not by $4$, which is also a contradiction.

Since both cases lead to contradictions, our initial assumption that there exists a natural number $k$ such that $5^k - 1$ is not divisible by $4$ is false. Therefore, for all natural numbers $n \geq 1$, $4$ divides $(5^n - 1)$.

Hence, the statement has been proven by the method of smallest counterexample.
\end{solution}



\begin{question}
    Let $n\in \Z$. Use induction to prove that $3 \mid (n^3+2n)$. 
\end{question}
% Student: put your answer between the next two lines.
\begin{solution}
We will prove the statement by mathematical induction.

\textbf{Base Case:} Let $n = 0$. We need to show that $3$ divides $(0^3 + 2 \cdot 0)$, which simplifies to $0$. Since $0$ is divisible by $3$ (as $0 = 3 \cdot 0$), the base case is true.

\textbf{Inductive Hypothesis:} Assume that for some positive integer $k$, $3$ divides $(k^3 + 2k)$, i.e., $k^3 + 2k = 3m$ for some integer $m$.

\textbf{Inductive Step:} We need to show that $3$ divides $((k + 1)^3 + 2(k + 1))$. Let's expand $((k + 1)^3 + 2(k + 1))$:

\[
((k + 1)^3 + 2(k + 1)) = (k^3 + 3k^2 + 3k + 1) + (2k + 2) = (k^3 + 2k) + 3(k^2 + k + 1).
\]

We already know that $3$ divides $(k^3 + 2k)$ by our inductive hypothesis (i.e., $k^3 + 2k = 3m$ for some integer $m$), so we have:

\[
(k^3 + 2k) + 3(k^2 + k + 1) = 3m + 3(k^2 + k + 1) = 3(m + k^2 + k + 1).
\]

Since $m$, $k^2$, $k$, and $1$ are all integers, their sum is also an integer. Let $n = m + k^2 + k + 1$, so $3(m + k^2 + k + 1) = 3n$. Thus, $3$ divides $((k + 1)^3 + 2(k + 1))$.

By the principle of mathematical induction, we have shown that for all positive integers $n$, $3$ divides $(n^3 + 2n)$.

Therefore, the statement has been proven by mathematical induction.
\end{solution}









\end{document}
