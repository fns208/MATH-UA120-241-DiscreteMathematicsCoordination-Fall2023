\documentclass{article}
% The LaTeX macro language is complicated, so we have inserted
% lots of documenting comments into the file.  Comments start
% with `%' and continue to the end of the line.  In Overleaf's
% window, they are colored green
%
% Comments prefixed with `Student:' are relevant to students.
% Skip anything else you don't understand, or ask me.
%
% set font encoding for PDFLaTeX or XeLaTeX
\usepackage{ifxetex}
\ifxetex
  \usepackage{fontspec}
\else
  \usepackage[T1]{fontenc}
  \usepackage[utf8]{inputenc}
  \usepackage{lmodern}
\fi

% Student: These lines describe some document metadata.
\title{Problem Set 5}
\author{%
% Student: change the next line to your name!
    Name
\\  MATH-UA 120 Discrete Mathematics
}
\date{due October 27, 2023}


\usepackage[headings=runin-fixed-nr]{exsheets}
% These make enumerates within questions start at the second ("(a)") level, rather than the first ("1.") level.
\makeatletter
    \newcommand{\stepenumdepth}{\advance\@enumdepth\@ne}
\makeatother
\SetupExSheets{
    question/pre-body-hook=\stepenumdepth,
    solution/pre-body-hook=\stepenumdepth,
}
\DeclareInstance{exsheets-heading}{runin-nn-np}{default}{
    runin = true,
    title-post-code = .\space,
    join = {
        main[r,vc]title[l,vc](0pt,0pt);
    }
}
\newif\ifshowsolutions
% Student: replace `false' with `true' to typeset your solutions.
% Otherwise they are ignored!
\showsolutionstrue
\ifshowsolutions
    \SetupExSheets{
        question/pre-hook=\itshape,
        solution/headings=runin-nn-np,
        solution/print=true,
        solution/name=Answer
    }%
    \makeatletter%
    \pretocmd{\@title}{Answers to }%
    \makeatother%
\else
    \SetupExSheets{solution/print=false}
\fi

% Bug workaround: http://tex.stackexchange.com/a/146536/1402
%\newenvironment{exercise}{}{}
\RenewQuSolPair{question}{solution}
%\let\answer\solution
%\let\endanswer\endsolution
\usepackage{manfnt}
\newcommand{\danger}{\marginpar[\hfill\dbend]{\dbend\hfill}}

% We are creating a command for some common commands.
\newcommand{\Z}{\mathbb{Z}}

% This package is for specifying graphics.  It's amazing.
% Manual at http://texdoc.net/texmf-dist/doc/generic/pgf/pgfmanual.pdf
\usepackage{tikz}

\usepackage{amsmath, amsthm, amssymb}
\usepackage{amsfonts}
\usepackage{siunitx}
\DeclareSIUnit\pound{lb}
\usepackage{hyperref}
\newtheorem*{theorem}{Theorem}
\theoremstyle{definition}
\newtheorem*{definition}{Definition}
% This is the beginning of the part of the file that describes
% the text of the document.
% That's why it says `\begin{document}' below. :-)
\begin{document}
\maketitle



These are to be written up in \LaTeX{} and turned in to Gradescope.\\



\ifshowsolutions
    \SetupExSheets{solution/print=true}
\else
    \danger
 \underline{ \LaTeX{}  Instructions:}  You can view the source (\texttt{.tex}) file to get some more examples of \LaTeX{} code.  I have commented the source file in places where new \LaTeX{} constructions are used.
  
  Remember to change \verb|\showsolutionsfalse| to \verb|\showsolutionstrue|
    in the document's preamble 
    (between \verb|\documentclass{article}| and \verb|\begin{document}|)
\fi

\section*{Assigned Problems}

\begin{question}
    Prove the following statement by contrapositive: \\
    For all $n\in \mathbb{N}$, if $2^n<n!$, then $n>3$.
\end{question}
% Student: put your answer between the next two lines.
\begin{solution}
We will prove the contrapositive of the statement; that is, for all $n\in \mathbb{N}$, if $n\leq 3$, then $2^n\geq n!$. Since there are only four natural numbers where $n\leq 3$, we only need to prove for the case when $n=0, 1, 2, 3$.
\begin{itemize}
\item When $n=0, 1 = 2^0 \geq 0!=1.$
\item When $n=1, 2 = 2^1 \geq 1!=1.$
\item When $n=2, 4 = 2^2 \geq 2!=2.$
\item When $n=3, 8 = 2^3 \geq 3!=6.$
\end{itemize}
Since the contrapositive is true, the statement holds.

{\color{red} Rubric:
\begin{itemize}
\item Follow RVF rubric with 1P for \LaTeX
\end{itemize}}
\end{solution}


\begin{question}
    Prove the following by contradiction:\\
    Let $A, B, C$ be sets. If $A\subseteq B$ and $B\cap C=\emptyset$, then $A\cap C=\emptyset$.
\end{question}
% Student: put your answer between the next two lines.
\begin{solution}
      Let $A, B, C$ be sets. Assume $A\subseteq B$ and $B\cap C=\emptyset$. Suppose, on the contrary, $A\cap C\neq\emptyset$. Let $x\in A\cap C$. Then $x\in A$ and $x\in C$. Since $A\subseteq B$, $x\in B$. We have $x\in B$ and $x\in C$, which implies $x \in B\cap C$. This contradicts $B\cap C=\emptyset$. Therefore, $A\cap C=\emptyset$.
	
{\color{red} Rubric:
\begin{itemize}
\item Follow RVF rubric with 1P for \LaTeX
\end{itemize}}
\end{solution}

\begin{question}
    Prove the following statement by contradiction:\\
    Let $x, y\in \Z$. Then $x^2-4y-3\neq 0$.
\end{question}
% Student: put your answer between the next two lines.
\begin{solution}
Suppose, for the sake of contradiction, that there are some integers $x$ and $y$ such that $x^2 - 4y - 3 = 0$. That is,
	\[ x^2 = 4y+3 = 2(2y+1)+1, \]
which means $x^2$ is odd. Then there exists $a\in \Z$ such that $x^2=2a+1$. Observe 
\begin{align*}
x^2 - 4y - 3 & = 0\\
(2a+1)^2 -4y -3 &= 0\\
4a^2+4a+1 -4y -3 & = 0\\
4a^2+4a+1 -4y & = 2\\
2(2a^2+2a - 2y) & = 2\\
2a^2+2a - 2y & = 1\\
2(a^2+a - y) & = 1.
\end{align*}
This implies that 1 is equal to an even number, contradiction! Therefore, $x^2-4y-3\neq 0$.
{\color{red} Rubric:
\begin{itemize}
\item Follow RVF rubric with 1P for \LaTeX
\end{itemize}}
\end{solution}

\begin{question}
    Prove the following by smallest counterexample:\\
    Let $n\in \mathbb{N}$. If $n\geq 1$, then $4 \mid (5^n-1)$.
\end{question}
% Student: put your answer between the next two lines.
\begin{solution}
      For the sake of contradiction, suppose that $4 \nmid (5^n-1)$ for $n\geq 1$. By the Well-Ordering Principle, there exists a smallest element $x\in \mathbb{N}$ such that $x\geq 1$ and $4 \nmid (5^x-1)$. We know that $x\neq 1$ because $5^1-1=4$ and $4\mid 4$. So $x\geq 2$. Note $x-1\in \mathbb{N}$ and $x-1\geq 1$. Then $4\mid (5^{x-1}-1)$, meaning there exists $a\in \Z$ such that $5^{x-1}-1 = 4a$. Observe
      \begin{align*}
      5^{x-1}-1 &= 4a\\
      5(5^{x-1}-1) &= 5(4a)\\
      5^x - 5 &= 20a\\
      5^x -1 &= 20a +4 = 4(5a+1).
      \end{align*}
      Then $4\mid (5^x-1)$, which is a contradiction. Therefore, $4 \mid (5^n-1)$ for $n\geq 1$.
{\color{red} Rubric:
\begin{itemize}
\item Follow RVF rubric with 1P for \LaTeX
\end{itemize}}
\end{solution}


\begin{question}
    Let $n\in \Z$. Use induction to prove there are $3 \mid (n^3+2n)$. 
\end{question}
% Student: put your answer between the next two lines.
\begin{solution}
        We will consider two cases: $n\geq 0$ and $n<0$.
    \begin{itemize}
        \item[Case 1:] Let $n\geq 0$. We will prove $3 \mid (n^3+2n)$ by induction.
	\begin{description}
	\item[Base Cases: ] Consider $n=0$. Since $0=3(0)$, $0^3+2(0)=0$ is divisible by 3.
	
	\item[Inductive Hypothesis: ] Consider $n=k$ for some $k\geq 0$.\\ Assume $3 \mid (k^3+2k)$.
	
	\item[Inductive Step: ] Consider $n=k+1$. We want to show $3\mid [(k+1)^3+2(k+1)]$. Since $3 \mid (k^3+2k)$, there exists $a\in \Z$ such that $k^3+2k=3a$. Observe
 \begin{align*}
     (k+1)^3 + 2(k+1) &= k^3+3k^2+3k+1 + 2k +2\\
     &= (k^3+2k) + (3k^2+3k+3)\\
     &= 3a + 3k^2+3k+3\\
     & = 3 (a + k^2+k+1).
 \end{align*}
 Since $a + k^2+k+1\in \Z$, $3\mid [(k+1)^3+2(k+1)]$.
	\end{description}
	Therefore, by the principle of mathematical induction, the statement is true for $n\geq 0$.

 \item[Case 2:] Let $n<0$. Then $-n>0$. We know $3 \mid [(-n)^3+2(-n)]$. Then there exists $b\in \Z$ such that $-n^3-2n=3b$. So 
 \[
 n^3+2n = -(3b) = 3 (-b).
 \]
 Since $-b\in\Z$, $3\mid (n^3+2n)$.

 \end{itemize}

 Therefore, we have proven for any integer $n$, $n^3+2n$ is divisible by 3.
	
{\color{red} Rubric:
\begin{itemize}
\item Follow RVF rubric with 1P for \LaTeX
\end{itemize}}
\end{solution}








\end{document}
