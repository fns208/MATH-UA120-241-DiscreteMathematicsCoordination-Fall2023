\documentclass{article}
% This is a LaTeX file.  It is a text file that is compiled
% by a program called LaTeX into a pretty PDF file.  
% If you're viewing this file on Overleaf, you'll see that PDF 
% in the window to the right.
%
% The LaTeX macro language is complicated, so we have inserted
% lots of documenting comments into the file.  Comments start
% with `%' and continue to the end of the line.  In Overleaf's
% window, they are colored green.
%
% Comments prefixed with `Student:' are relevant to students.
% Skip anything else you don't understand, or ask me.
%
% set font encoding for PDFLaTeX or XeLaTeX
\usepackage{ifxetex}
\ifxetex
  \usepackage{fontspec}
\else
  \usepackage[T1]{fontenc}
  \usepackage[utf8]{inputenc}
  \usepackage{lmodern}
\fi

% Student: These lines describe some document metadata.
\title{Problem Set 3}
\author{%
% Student: change the next line to your name!
    Name
\\  MATH-UA 120 Discrete Mathematics
}
\date{due October 6, 2023}


\usepackage[headings=runin-fixed-nr]{exsheets}
% These make enumerates within questions start at the second ("(a)") level, rather than the first ("1.") level.
\makeatletter
    \newcommand{\stepenumdepth}{\advance\@enumdepth\@ne}
\makeatother
\SetupExSheets{
    question/pre-body-hook=\stepenumdepth,
    solution/pre-body-hook=\stepenumdepth,
}
\DeclareInstance{exsheets-heading}{runin-nn-np}{default}{
    runin = true,
    title-post-code = .\space,
    join = {
        main[r,vc]title[l,vc](0pt,0pt);
    }
}
\newif\ifshowsolutions
% Student: replace `false' with `true' to typeset your solutions.
% Otherwise they are ignored!
\showsolutionstrue
\ifshowsolutions
    \SetupExSheets{
        question/pre-hook=\itshape,
        solution/headings=runin-nn-np,
        solution/print=true,
        solution/name=Answer
    }%
    \makeatletter%
    \pretocmd{\@title}{Answers to }%
    \makeatother%
\else
    \SetupExSheets{solution/print=false}
\fi

% Bug workaround: http://tex.stackexchange.com/a/146536/1402
%\newenvironment{exercise}{}{}
\RenewQuSolPair{question}{solution}
%\let\answer\solution
%\let\endanswer\endsolution
\usepackage{manfnt}
\newcommand{\danger}{\marginpar[\hfill\dbend]{\dbend\hfill}}

\usepackage{subcaption}

\newcommand{\Z}{\mathbb{Z}}
\newcommand{\R}{\mathbb{R}}
\newcommand{\N}{\mathbb{N}}
\newcommand{\Q}{\mathbb{Q}}

\usepackage{amsmath, amsthm}
\usepackage{amsfonts}
\usepackage{enumerate}
\usepackage{siunitx}
\DeclareSIUnit\pound{lb}
\usepackage{hyperref}
\newtheorem*{theorem}{Theorem}
\newtheorem*{proposition}{Proposition}
\newtheorem*{claim}{Claim}
\theoremstyle{definition}
\newtheorem*{definition}{Definition}
% This is the beginning of the part of the file that describes
% the text of the document.
% That's why it says `\begin{document}' below. :-)
\begin{document}
\maketitle



These are to be written up and turned in to Gradescope.\\



\ifshowsolutions
    \SetupExSheets{solution/print=true}
\else
    \danger
 \underline{ \LaTeX  Instructions:}  You can view the source (\texttt{.tex}) file to get some more examples of \LaTeX{} code.  I have commented the source file in places where new \LaTeX{} constructions are used.
  
  Remember to change \verb|\showsolutionsfalse| to \verb|\showsolutionstrue|
    in the document's preamble 
    (between \verb|\documentclass{article}| and \verb|\begin{document}|)
\fi

\section*{Assigned Problems}

\begin{question}
   \begin{enumerate}
   \item Consider the following subsets of $\N$.
       \begin{align*}
           A &= \text{The set of all even numbers.}\\
           B &= \text{The set of all prime numbers.}\\
           C &= \text{The set of all perfect squares.}\\
           D &= \text{The set of all multiples of 10.}
       \end{align*}
       Using \textbf{only} the symbols $3, A, B, C, D, \N, \in, \subseteq, =, \neq, \cap, \cup, \times, -, \emptyset$, ``('', and  ``)'', rewrite the following statements in set notation. 
           \begin{enumerate}
               \item None of the perfect squares are prime numbers. 
               \item The number 3 is a prime number that is not even.
               \item If you take all the prime numbers, all the even numbers, all the perfect squares, and all the multiples of 10, you still won't have all the natural numbers.
           \end{enumerate}
   
   \item Consider the following subsets of the set of all students at some university. 
       \begin{align*}
           F &= \text{The set of all freshmen.}\\
           S &= \text{The set of all seniors.}\\
           M &= \text{The set of all math majors.}\\
           C &= \text{The set of all CS majors.}
       \end{align*}
           \begin{enumerate}
               \item Using only the symbols $F, S, M, C, | ~ |, \cap, \cup, -$, and $>$, translate the following statement into the language of set theory. 
               \begin{quote}
                   ``There are more freshmen who aren't math majors than there are senior CS majors.''
               \end{quote}
               \item Translate the following statement in set theory into everyday English. 
               $$(F\cap M)\subseteq C$$
           \end{enumerate}
   \end{enumerate}
\end{question}
% Student: put your answer between the next two lines.
\begin{solution}
\begin{enumerate}
    \item 
   \begin{enumerate}
       \item Using set notation, we can rewrite the given statement as:
       \[
       \text{None of the perfect squares are prime numbers.} \quad \text{i.e.,} \quad C \cap B = \emptyset
       \]
       
       \item The number 3 is a prime number that is not even can be represented as:
       \[
       3 \in B \quad \text{and} \quad 3 \notin A
       \]
       
       \item The given statement in set notation is:
       \[
       (B \cup A \cup C \cup D) \neq \N
       \]
   \end{enumerate}
   
  \item \begin{enumerate}
       \item Using the given symbols, the statement can be translated as:
       \[
       |(F \cap C')| > |(S \cap C)|
       \]
       where $C'$ is the complement of set $C$.
       
       \item In everyday English, the statement translates to:
       "The intersection of the set of freshmen and math majors is a subset of the set of CS majors."
   \end{enumerate}
   \end{enumerate}
\end{solution}


\begin{question}
Describe explicitly in English the following sets, then give their cardinality.

\begin{enumerate}
	\item $\{x \in 2^{\Z} : 5 \in x \}$
	\item $\{x \in 2^{\Z} : x \subseteq \{ 1, 2, 3\} \}$
	\item $\{x \in 2^{\Z} : x \subseteq \{ 1, 2, \{3, 4\} \} \}$
	\item $\{x \in 2^{\Z} : x \in \{ 1, 2, \{3, 4\} \} \}$
	\item $\{x \in 2^{\Z} : y \in x \implies y = 0 \}$
\end{enumerate}
\end{question}
% Student: put your answer between the next two lines.
\begin{solution}
\begin{enumerate}
    \item $\{x \in 2^{\Z} : 5 \in x \}$ represents the set of all subsets of integers that contain 5. In English, it's the set of all subsets of integers that include the number 5.
    
    \item $\{x \in 2^{\Z} : x \subseteq \{ 1, 2, 3\} \}$ is the set of all subsets of integers that are entirely contained within the set $\{1, 2, 3\}$.
    
    \item $\{x \in 2^{\Z} : x \subseteq \{ 1, 2, \{3, 4\} \} \}$ represents the set of all subsets of integers that are entirely contained within the set $\{1, 2, \{3, 4\}\}$.
    
    \item $\{x \in 2^{\Z} : x \in \{ 1, 2, \{3, 4\} \} \}$ is not a valid set notation. It seems to be trying to represent the set of all subsets that are equal to $\{1, 2, \{3, 4\}\}$, but the notation is incorrect.
    
    \item $\{x \in 2^{\Z} : y \in x \implies y = 0 \}$ represents the set of all subsets of integers where every element is 0.
\end{enumerate}
\end{solution}



\begin{question}
\begin{enumerate}
	\item For each of the following statements, describe it in English, and say if it is true or false (without proof). Then write its negation using quantifiers, and express this negation in English. For instance, the statement $\forall x \in \Z \; x < 0$ means every integer is negative, and it is false. Its negation is $\exists x \in \Z \; x \geq 0$, which means that there exists a nonnegative integer.
	
	\begin{enumerate}
		\item $\forall x \in \Z \; \exists y \in \Z \; x^2 + y = 4$
		\item $\exists y \in \Z \; \forall x \in \Z \; x^2 + y = 4$
		\item $\forall n \in \Z \; \exists k \in \Z \; \exists d \in \Z \; k+ n = 2d$
		\item $\exists n \in \Z \; \forall k \in \Z \; \exists d \in \Z \; k+ n = 2d$
	\end{enumerate}
	
	\item For each of the statements (iii) and (iv): prove it if it is true, or prove the negation if it is false. These proofs are short.
\end{enumerate}
\end{question}
% Student: put your answer between the next two lines.
\begin{solution}
\begin{enumerate}
    \item 
\begin{enumerate}
    \item $\forall x \in \Z \; \exists y \in \Z \; x^2 + y = 4$ means for every integer $x$, there exists an integer $y$ such that $x^2 + y = 4$. This statement is true.
    \item $\exists y \in \Z \; \forall x \in \Z \; x^2 + y = 4$ means there exists an integer $y$ such that for every integer $x$, $x^2 + y = 4$. This statement is false.
    \item $\forall n \in \Z \; \exists k \in \Z \; \exists d \in \Z \; k+ n = 2d$ means for every integer $n$, there exist integers $k$ and $d$ such that $k + n = 2d$. This statement is true.
    \item $\exists n \in \Z \; \forall k \in \Z \; \exists d \in \Z \; k+ n = 2d$ means there exists an integer $n$ such that for every integer $k$, there exist integers $d$ such that $k + n = 2d$. This statement is false.
\end{enumerate}
    \item 
\begin{enumerate}
    \item[(iii)]
Let \( n \) be any integer. We want to show that there exist integers \( k \) and \( d \) such that \( k + n = 2d \).

Consider \( k = n \) and \( d = 0 \). Then, \( k + n = n + n = 2n \), where \( k \) and \( d \) are both integers. Thus, we have found integers \( k \) and \( d \) such that \( k + n = 2d \), as desired.

Therefore, for any integer \( n \), there exist integers \( k \) and \( d \) such that \( k + n = 2d \). This proves the statement \( \forall n \in \Z \; \exists k \in \Z \; \exists d \in \Z \; k+ n = 2d \).


    \item[(iv)]
The negation of the given statement \( \exists n \in \Z \; \forall k \in \Z \; \exists d \in \Z \; k + n = 2d \) is:
\[ \forall n \in \Z \; \exists k \in \Z \; \forall d \in \Z \; k + n \neq 2d \]

We want to prove this negation.

Assume \( n \) is any integer. We need to show that for this \( n \), there exists an integer \( k \) such that for all integers \( d \), \( k + n \) is not equal to \( 2d \).

Let \( k = 0 \). Now, consider any integer \( d \). We have \( k + n = 0 + n = n \) and \( 2d \) is an even number. Since \( n \) and \( 2d \) cannot be equal for any \( d \), we have shown that for this choice of \( k \), for all integers \( d \), \( k + n \) is not equal to \( 2d \).

Therefore, for any integer \( n \), there exists an integer \( k = 0 \) such that for all integers \( d \), \( k + n \) is not equal to \( 2d \). This proves the negation.


    \end{enumerate}
\end{enumerate}
\end{solution}




\begin{question}
   Let $I=\{1, 2, \dots, n\}$. Given a collection of sets $\{A_1,A_2,\dots, A_n\}$, denoted by $\{A_i\}_{i\in I}$. $\{A_i\}_{i\in I}$ is said to be \textbf{disjoint} if $\cap_{i\in I}A_i=\emptyset$, and it is said to be \textbf{pairwise disjoint} if $A_i\cap A_j=\emptyset$ whenever $i\neq j$. What is the difference between a \textbf{disjoint} collection of sets and a \textbf{pairwise disjoint} collection of sets? (\textit{Draw a picture to convince yourself. You do not need to submit this picture.}) Give an example of a collection of sets that is disjoint, but not pairwise disjoint. (\textit{Hint: You need a minimum of three sets.})
\end{question}
% Student: put your answer between the next two lines.
\begin{solution}
The difference between a \textbf{disjoint} collection of sets and a \textbf{pairwise disjoint} collection of sets is as follows:
A \textbf{disjoint} collection of sets means that the sets have no elements in common at all.
A \textbf{pairwise disjoint} collection of sets means that every pair of sets in the collection has no elements in common, but it is possible that some sets have elements in common with others.

For example:
Consider three sets: $A = \{1, 2\}$, $B = \{3, 4\}$, and $C = \{1, 3\}$. These sets are disjoint because no two sets have elements in common. However, they are not pairwise disjoint because $A$ and $C$ share the element 1.
\end{solution}



\begin{question}
   Let $E$ be the set of all even integers and let $O$ be the set of all odd integers. Let $X=\{ n\in \Z : n = x+y \text{ for some } x, y\in O\}$. Prove $X=E$.
\end{question}
% Student: put your answer between the next two lines.
\begin{solution}
To prove $X = E$, we need to show two things: $X \subseteq E$ and $E \subseteq X$.

First, let's show $X \subseteq E$. Let $x$ be an element of $X$. This means there exist odd numbers $a$ and $b$ such that $x = a + b$. Since the sum of two odd numbers is even, $x$ is even, and thus $x \in E$. Therefore, $X \subseteq E$.

Next, let's show $E \subseteq X$. Let $y$ be an element of $E$. This means $y$ is even, so $y = 2k$ for some integer $k$. Since $2k$ is the sum of two odd numbers (e.g., $k + (k-1)$), $y$ is in the set $X$. Therefore, $E \subseteq X$.

Thus, we have shown both $X \subseteq E$ and $E \subseteq X$, implying $X = E$.
\end{solution}



\end{document}