\documentclass{article}
% The LaTeX macro language is complicated, so we have inserted
% lots of documenting comments into the file.  Comments start
% with `%' and continue to the end of the line.  In Overleaf's
% text editing window, they are colored brownish-red.
%
% Comments prefixed with `Student:' are relevant to students.
% Skip anything else you don't understand, or ask your instructor.
%
% set font encoding for PDFLaTeX or XeLaTeX
\usepackage{ifxetex}
\ifxetex
  \usepackage{fontspec}
\else
  \usepackage[T1]{fontenc}
  \usepackage[utf8]{inputenc}
  \usepackage{lmodern}
\fi

% Student: These lines describe some document metadata.
\title{Problem Set 4}
\author{%
% Student: change the next line to your name!
    Name
\\  MATH-UA 120 Discrete Mathematics
}
\date{due October 13, 2023}


\usepackage[headings=runin-fixed-nr]{exsheets}
% These make enumerates within questions start at the second ("(a)") level, rather than the first ("1.") level.
\makeatletter
    \newcommand{\stepenumdepth}{\advance\@enumdepth\@ne}
\makeatother
\SetupExSheets{
    question/pre-body-hook=\stepenumdepth,
    solution/pre-body-hook=\stepenumdepth,
}
\DeclareInstance{exsheets-heading}{runin-nn-np}{default}{
    runin = true,
    title-post-code = .\space,
    join = {
        main[r,vc]title[l,vc](0pt,0pt);
    }
}
\newif\ifshowsolutions
% Student: replace `false' with `true' to typeset your solutions.
% Otherwise they are ignored!
\showsolutionstrue 
\ifshowsolutions
    \SetupExSheets{
        question/pre-hook=\itshape,
        solution/headings=runin-nn-np,
        solution/print=true,
        solution/name=Answer
    }%
    \makeatletter%
    \pretocmd{\@title}{Answers to }%
    \makeatother%
\else
    \SetupExSheets{solution/print=false}
\fi

% Bug workaround: http://tex.stackexchange.com/a/146536/1402
%\newenvironment{exercise}{}{}
\RenewQuSolPair{question}{solution}
%\let\answer\solution
%\let\endanswer\endsolution
\usepackage{manfnt}
\newcommand{\danger}{\marginpar[\hfill\dbend]{\dbend\hfill}}

% We are creating a command for some common commands.
\newcommand{\Z}{\mathbb{Z}}
\newcommand{\modulo}{\text{mod }}

% This package is for specifying graphics.  It's amazing.
% Manual at http://texdoc.net/texmf-dist/doc/generic/pgf/pgfmanual.pdf
\usepackage{tikz}

\usepackage{amsmath, amsthm}
\usepackage{amsfonts}
\usepackage{siunitx}
\DeclareSIUnit\pound{lb}
\usepackage{hyperref}
\newtheorem*{theorem}{Theorem}
\theoremstyle{definition}
\newtheorem*{definition}{Definition}
% This is the beginning of the part of the file that describes
% the text of the document.
% That's why it says `\begin{document}' below. :-)
\begin{document}
\maketitle



These are to be written up in \LaTeX{} and turned in to Gradescope.\\



\ifshowsolutions
    \SetupExSheets{solution/print=true}
\else
    \danger
 \underline{ \LaTeX  Instructions:}  You can view the source (\texttt{.tex}) file to get some more examples of \LaTeX{} code.  I have commented the source file in places where new \LaTeX{} constructions are used.
  
  Remember to change \verb|\showsolutionsfalse| to \verb|\showsolutionstrue|
    in the document's preamble 
    (between \verb|\documentclass{article}| and \verb|\begin{document}|)
\fi

\section*{Assigned Problems}

\begin{question}
    Consider the set $A = \{0, 1, 2, \dots, 8 \}$. Define a relation $R$ on $A$ by
	\[
	a\mathrel{R}b \iff a^2 \equiv b^2 \pmod{9}.
	\]
	\begin{enumerate}
	\item Show that $R$ is an equivalence relation, 
	\item then determine all its (distinct) equivalence classes.
	\end{enumerate}
\end{question}
% Student: put your answer between the next two lines.
\begin{solution}
\begin{enumerate}
    \item To show that \(R\) is an equivalence relation, we need to verify three properties: reflexivity, symmetry, and transitivity.
    
    - Reflexivity: For any \(a \in A\), \(a^2 \equiv a^2 \pmod{9}\), so \(a \mathrel{R} a\). Hence, \(R\) is reflexive.
    
    - Symmetry: If \(a \mathrel{R} b\), then \(a^2 \equiv b^2 \pmod{9}\). This implies \(b^2 \equiv a^2 \pmod{9}\), so \(b \mathrel{R} a\). Thus, \(R\) is symmetric.
    
    - Transitivity: If \(a \mathrel{R} b\) and \(b \mathrel{R} c\), then \(a^2 \equiv b^2 \pmod{9}\) and \(b^2 \equiv c^2 \pmod{9}\). Combining these congruences, we get \(a^2 \equiv c^2 \pmod{9}\), so \(a \mathrel{R} c\). Thus, \(R\) is transitive.
    
    Since \(R\) satisfies reflexivity, symmetry, and transitivity, it is an equivalence relation.
    
    \item To determine the equivalence classes, we'll consider the squares modulo 9 and group elements accordingly:
    \begin{itemize}
    \item  \( [0] = \{0, 9\} \)
    \item  \( [1] = \{1, 8\} \)
    \item  \( [4] = \{4, 5\} \)
    \item  \( [2] = \{2, 7\} \)
    \item  \( [8] = \{8, 1\} \)
    \item  \( [5] = \{5, 4\} \)
    \item  \( [7] = \{7, 2\} \)
    \item  \( [3] = \{3, 6\} \)
    \item  \( [6] = \{6, 3\} \)
    \end{itemize}
\end{enumerate}
\end{solution}



\begin{question}
    Are the given relations irreflexive? antisymmetric? transitive? Either \textit{prove} generally or \textit{disprove} via 
    counterexample.
    	\begin{enumerate}
	\item For $x, y \in \Z$,  $x\mathrel{R}y \iff |x - y| > 0$. 
	\item  $x\mathrel{R}y$ means that $x$ and $y$ have a common prime factor (a prime number that divides both $x$ and $y$), 
	where $x, y \in \Z$.
	\item For $x, y \in 2^{\Z}$. $x\mathrel{R}y \iff x \cap y \neq \emptyset$.
	\end{enumerate}
\end{question}
% Student: put your answer between the next two lines.
\begin{solution}
\begin{enumerate}
    \item 
    - \(x \mathrel{R} y \iff |x - y| > 0\)
    
    This relation is not irreflexive since, for any \(x \in \Z\), \(|x - x| = 0\) is not greater than 0.
    
    This relation is antisymmetric since if \(|x - y| > 0\) and \(|y - x| > 0\), then \(x \neq y\) and \(y \neq x\), implying \(x \neq y\), which is the definition of antisymmetry.
    
    This relation is not transitive. For example, let \(x = 1\), \(y = 2\), and \(z = 3\). We have \(|x - y| = 1\) and \(|y - z| = 1\), but \(|x - z| = 2\), violating transitivity.
    
    \item 
    - \(x \mathrel{R} y\) means that \(x\) and \(y\) have a common prime factor.
    
    This relation is not irreflexive since, for any \(x \in \Z\), \(x\) has a common prime factor with itself (itself being the prime factor).
    
    This relation is antisymmetric since if \(x\) and \(y\) have a common prime factor, then \(y\) and \(x\) also have a common prime factor, and this is a symmetric relation. Since the only way this can happen is if \(x = y\), the relation is antisymmetric.
    
    This relation is transitive. If \(x\) has a common prime factor with \(y\) and \(y\) has a common prime factor with \(z\), then \(x\) has a common prime factor with \(z\), which is the definition of transitivity for this relation.
    
    \item 
    - \(x \mathrel{R} y \iff x \cap y \neq \emptyset\)
    
    This relation is not irreflexive since any nonempty set intersects itself, making the relation reflexive.
    
    This relation is not antisymmetric since, for example, if \(x = \{1, 2\}\) and \(y = \{2, 3\}\), both \(x \cap y\) and \(y \cap x\) are nonempty, so \(x \mathrel{R} y\) and \(y \mathrel{R} x\), violating antisymmetry.
    
    This relation is transitive. If \(x \mathrel{R} y\) and \(y \mathrel{R} z\), it means \(x \cap y \neq \emptyset\) and \(y \cap z \neq \emptyset\). Then, \(x \cap y \cap z \neq \emptyset\), implying \(x \mathrel{R} z\), which is the definition of transitivity.
\end{enumerate}
\end{solution}


\begin{question}
    \begin{enumerate}
   	\item What is the total number of partitions in two of $\{1, 2, \dots, 100 \}$? 
	Remember, both parts should be non-empty.
        \item Suppose that a single character is stored in a computer using eight bits. 
        How many bit patterns have at least two 1's?
   	\end{enumerate}
\end{question}
% Student: put your answer between the next two lines.
\begin{solution}
\begin{enumerate}
    \item To find the total number of partitions of the set $\{1, 2, \dots, 100\}$ into two non-empty parts, we can use combinatorial reasoning. Each element can either be in the first part or the second part, giving us $2^{100}$ ways to distribute them. However, we must subtract the cases where one of the parts is empty, which is choosing one of the $2$ sets to be empty and then subtracting $1$ for the case where both are empty. So, the total number of partitions with non-empty parts is $2^{100} - 2 = 2^{100} - 2$.
    
    \item A single character stored in a computer using eight bits can have $2^8 = 256$ possible bit patterns. However, we need to exclude the patterns with no 1's and patterns with exactly one 1 (as we want at least two 1's). 
    
    - Patterns with no 1's: There is only one pattern, which is all $0$'s.
    - Patterns with exactly one 1: There are $8$ ways to place the single 1 in one of the $8$ positions.
    
    So, the total number of bit patterns with at least two 1's is $256 - 1 - 8 = 247$.
\end{enumerate}
\end{solution}



\begin{question}
    \begin{enumerate}
	\item Twenty people are to be divided into two teams with ten players on each team.  
	In how many ways can this be done?
        \item Thirty five discrete math students are to be divided into seven discussion groups, each consisting of five students.  
        In how many ways can this be done?
   	\end{enumerate}
\end{question}
% Student: put your answer between the next two lines.
\begin{solution}
\begin{enumerate}
    \item 
    - To partition 20 people into two teams of 10 each, we first choose 10 people for the first team, and the rest automatically form the second team. The number of ways to choose 10 people from 20 is given by the binomial coefficient \( \binom{20}{10} \):
    \[ \binom{20}{10} = \frac{20!}{10! \times 10!} = 184,756. \]
    
    \item 
    - To divide 35 students into 7 groups of 5 each, we again use binomial coefficients. We choose 5 students for the first group, then 5 from the remaining for the second, and so on. The number of ways is given by the product of binomial coefficients:
    \[ \binom{35}{5} \times \binom{30}{5} \times \ldots \times \binom{5}{5} = \frac{35!}{5! \times 5! \times \ldots \times 5!} = 7! = 5,040. \]
\end{enumerate}
\end{solution}




\begin{question}
    Prove \textbf{combinatorially} that
    \[ 3^n = \binom{n}{0} \cdot 2^0 + \binom{n}{1}\cdot 2^1+ \binom{n}{2}\cdot 2^2+\cdots +\binom{n}{n}\cdot 2^n. \]
    \textbf{You may not manipulate the question algebraically.}
\end{question}
% Student: put your answer between the next two lines.
\begin{solution}
To prove this combinatorially, we consider the binomial theorem, which states:
\[ (a + b)^n = \binom{n}{0} \cdot a^n \cdot b^0 + \binom{n}{1} \cdot a^{n-1} \cdot b^1 + \ldots + \binom{n}{n} \cdot a^0 \cdot b^n. \]
Here, we let \(a = 1\) and \(b = 2\), so the left-hand side becomes \(3^n\). We get:
\[ 3^n = \binom{n}{0} \cdot 1^n \cdot 2^0 + \binom{n}{1} \cdot 1^{n-1} \cdot 2^1 + \ldots + \binom{n}{n} \cdot 1^0 \cdot 2^n. \]
Simplifying, we obtain the desired equation:
\[ 3^n = \binom{n}{0} \cdot 2^0 + \binom{n}{1} \cdot 2^1 + \ldots + \binom{n}{n} \cdot 2^n. \]
\end{solution}



\end{document}
