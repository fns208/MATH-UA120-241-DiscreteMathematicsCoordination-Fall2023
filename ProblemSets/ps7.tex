\documentclass{article}
% The LaTeX macro language is complicated, so we have inserted
% lots of documenting comments into the file.  Comments start
% with `%' and continue to the end of the line.  In Overleaf's
% window, they are colored green
%
% Comments prefixed with `Student:' are relevant to students.
% Skip anything else you don't understand, or ask me.
%
% set font encoding for PDFLaTeX or XeLaTeX
\usepackage{ifxetex}
\ifxetex
  \usepackage{fontspec}
\else
  \usepackage[T1]{fontenc}
  \usepackage[utf8]{inputenc}
  \usepackage{lmodern}
\fi

% Student: These lines describe some document metadata.
\title{Problem Set 7}
\author{%
% Student: change the next line to your name!
    Name
\\  MATH-UA 120 Discrete Mathematics
}
\date{due December 1, 2023}


\usepackage[headings=runin-fixed-nr]{exsheets}
% These make enumerates within questions start at the second ("(a)") level, rather than the first ("1.") level.
\makeatletter
    \newcommand{\stepenumdepth}{\advance\@enumdepth\@ne}
\makeatother
\SetupExSheets{
    question/pre-body-hook=\stepenumdepth,
    solution/pre-body-hook=\stepenumdepth,
}
\DeclareInstance{exsheets-heading}{runin-nn-np}{default}{
    runin = true,
    title-post-code = .\space,
    join = {
        main[r,vc]title[l,vc](0pt,0pt);
    }
}
\newif\ifshowsolutions
% Student: replace `false' with `true' to typeset your solutions.
% Otherwise they are ignored!
\showsolutionsfalse
\ifshowsolutions
    \SetupExSheets{
        question/pre-hook=\itshape,
        solution/headings=runin-nn-np,
        solution/print=true,
        solution/name=Answer
    }%
    \makeatletter%
    \pretocmd{\@title}{Answers to }%
    \makeatother%
\else
    \SetupExSheets{solution/print=false}
\fi

% Bug workaround: http://tex.stackexchange.com/a/146536/1402
%\newenvironment{exercise}{}{}
\RenewQuSolPair{question}{solution}
%\let\answer\solution
%\let\endanswer\endsolution
\usepackage{manfnt}
\newcommand{\danger}{\marginpar[\hfill\dbend]{\dbend\hfill}}

% We are creating a command for some common commands.
\newcommand{\Z}{\mathbb{Z}}
\newcommand{\N}{\mathbb{N}}
\newcommand{\im}{\operatorname{im}}
\newcommand{\id}{\operatorname{id}}

% This package is for specifying graphics.  It's amazing.
% Manual at http://texdoc.net/texmf-dist/doc/generic/pgf/pgfmanual.pdf
\usepackage{tikz}

\usepackage{amsmath, amsthm}
\usepackage{amsfonts}
\usepackage{siunitx}
\DeclareSIUnit\pound{lb}
\usepackage{hyperref}
\newtheorem*{theorem}{Theorem}
\theoremstyle{definition}
\newtheorem*{definition}{Definition}
% This is the beginning of the part of the file that describes
% the text of the document.
% That's why it says `\begin{document}' below. :-)
\begin{document}
\maketitle



These are to be written up in \LaTeX{} and turned in to Gradescope.\\



\ifshowsolutions
    \SetupExSheets{solution/print=true}
\else
    \danger
 \underline{ \LaTeX{}  Instructions:}  You can view the source (\texttt{.tex}) file to get some more examples of \LaTeX{} code.  I have commented the source file in places where new \LaTeX{} constructions are used.
  
  Remember to change \verb|\showsolutionsfalse| to \verb|\showsolutionstrue|
    in the document's preamble 
    (between \verb|\documentclass{article}| and \verb|\begin{document}|)
\fi

\section*{Assigned Problems}



\begin{question}
        A fair coin is flipped 10 times.
    \begin{enumerate}
        \item What is the probability that there are an equal number of head and
            tails?
        \item What is the probability the first three flips are heads?
        \item What is the probability that there are an equal number of heads
            and tails and the first three flips are heads?
        \item What is the probability that there are an equal number of heads
            and tails or the first three flips are heads (or both)?
        \item What is the probability that the first three flips are heads given
            that an equal number of heads and tails are flipped?
    \end{enumerate}
\end{question}
% Student: put your answer between the next two lines.
\begin{solution}
\end{solution}

\begin{question}
    An unfair coin shows heads with probability $p$ and tails with probability
    $1-p$. Suppose this coin is flipped $2$ times. Let $A$ be the event that the coin comes up first heads and
    then tails.  Let $B$ be the event that the coin comes up first tails and
    then heads.
    \begin{enumerate}
        \item  Find $P(A)$.
        \item  Find $P(B)$.
        \item  Find $P(A \mid A \cup B)$.
        \item  Find $P(B \mid A \cup B)$.
    \end{enumerate}
    Explain how one could use this unfair coin to make a fair decision.
\end{question}
% Student: put your answer between the next two lines.
\begin{solution}
\end{solution}

\begin{question}
    Suppose that $A$ and $B$ are events in a sample space $(S,P)$.
    Prove or disprove:
    \begin{enumerate}
        \item If $P(A \cap B)=0$, then $P(A\mid B)=P(B\mid A)$ if and only
            if $P(A)=P(B)$.
        \item If $P(A)>0, P(B)>0$ but $P(A \cap B)=0$, then $P(A\mid
            B)=P(B\mid A)$.  If proven, give an example of two such events
            with $P(A) \ne P(B)$.
    \end{enumerate}
\end{question}
% Student: put your answer between the next two lines.
\begin{solution}
\end{solution}

\begin{question}
    Consider the sample space $S = \{ a, b, c \}$, with equal probability for each outcome. Define the random variables $X$ and  
    $Y$ by $X(a) = -1, \quad X(b) = 0, \quad X(c) = 1, \quad Y(a) = Y(c) = 0, \quad Y(b) = 1.$ 
    Check that $\mathrm{Var}(X+Y) = \mathrm{Var}(X) + \mathrm{Var}(Y)$, but that $X$ and $Y$ are not independent.
\end{question}
% Student: put your answer between the next two lines.
\begin{solution}
\end{solution}

\begin{question}
    Provide an alternative proof to Proposition 31.7 using any of the statements in Proposition 31.8.
    \begin{theorem}[Proposition 31.7]
    Let $A$ and $B$ be events in a sample space $(S, P)$. Then 
    \[ P(A) + P(B) = P(A\cup B) + P(A\cap B).\]
    \end{theorem}
\end{question}
% Student: put your answer between the next two lines.
\begin{solution}
\end{solution}

\end{document}
