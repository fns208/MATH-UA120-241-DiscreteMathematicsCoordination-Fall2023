\documentclass{article}
% The LaTeX macro language is complicated, so we have inserted
% lots of documenting comments into the file.  Comments start
% with `%' and continue to the end of the line.  In Overleaf's
% window, they are colored green
%
% Comments prefixed with `Student:' are relevant to students.
% Skip anything else you don't understand, or ask me.
%
% set font encoding for PDFLaTeX or XeLaTeX
\usepackage{ifxetex}
\ifxetex
  \usepackage{fontspec}
\else
  \usepackage[T1]{fontenc}
  \usepackage[utf8]{inputenc}
  \usepackage{lmodern}
\fi

% Student: These lines describe some document metadata.
\title{Problem Set 6}
\author{%
% Student: change the next line to your name!
    Name
\\  MATH-UA 120 Discrete Mathematics
}
\date{due November 10, 2023}


\usepackage[headings=runin-fixed-nr]{exsheets}
% These make enumerates within questions start at the second ("(a)") level, rather than the first ("1.") level.
\makeatletter
    \newcommand{\stepenumdepth}{\advance\@enumdepth\@ne}
\makeatother
\SetupExSheets{
    question/pre-body-hook=\stepenumdepth,
    solution/pre-body-hook=\stepenumdepth,
}
\DeclareInstance{exsheets-heading}{runin-nn-np}{default}{
    runin = true,
    title-post-code = .\space,
    join = {
        main[r,vc]title[l,vc](0pt,0pt);
    }
}
\newif\ifshowsolutions
% Student: replace `false' with `true' to typeset your solutions.
% Otherwise they are ignored!
\showsolutionstrue
\ifshowsolutions
    \SetupExSheets{
        question/pre-hook=\itshape,
        solution/headings=runin-nn-np,
        solution/print=true,
        solution/name=Answer
    }%
    \makeatletter%
    \pretocmd{\@title}{Answers to }%
    \makeatother%
\else
    \SetupExSheets{solution/print=false}
\fi

% Bug workaround: http://tex.stackexchange.com/a/146536/1402
%\newenvironment{exercise}{}{}
\RenewQuSolPair{question}{solution}
%\let\answer\solution
%\let\endanswer\endsolution
\usepackage{manfnt}
\newcommand{\danger}{\marginpar[\hfill\dbend]{\dbend\hfill}}

% We are creating a command for some common commands.
\newcommand{\Z}{\mathbb{Z}}
\newcommand{\N}{\mathbb{N}}
\newcommand{\R}{\mathbb{R}}
\newcommand{\im}{\operatorname{im}}
\newcommand{\id}{\operatorname{id}}

% This package is for specifying graphics.  It's amazing.
% Manual at http://texdoc.net/texmf-dist/doc/generic/pgf/pgfmanual.pdf
\usepackage{tikz}

\usepackage{amsmath, amsthm}
\usepackage{amsfonts}
\usepackage{siunitx}
\DeclareSIUnit\pound{lb}
\usepackage{hyperref}
\newtheorem*{theorem}{Theorem}
\theoremstyle{definition}
\newtheorem*{definition}{Definition}
% This is the beginning of the part of the file that describes
% the text of the document.
% That's why it says `\begin{document}' below. :-)
\begin{document}
\maketitle



These are to be written up in \LaTeX{} and turned in to Gradescope.\\



\ifshowsolutions
    \SetupExSheets{solution/print=true}
\else
    \danger
 \underline{ \LaTeX{}  Instructions:}  You can view the source (\texttt{.tex}) file to get some more examples of \LaTeX{} code.  I have commented the source file in places where new \LaTeX{} constructions are used.
  
  Remember to change \verb|\showsolutionsfalse| to \verb|\showsolutionstrue|
    in the document's preamble 
    (between \verb|\documentclass{article}| and \verb|\begin{document}|)
\fi

\section*{Assigned Problems}


\begin{question}
    Suppose that we have two piles of cards each containing $n$ cards. Two players play a game as follows. Each player, in turn, chooses one pile and then removes any number of cards, but at least one, from the chosen pile. The player who removes the last card wins the game. Show that the second player can always win the game. 
    \textit{Hint: Use strong induction.}
\end{question}
% Student: put your answer between the next two lines.
\begin{solution}
We will prove by strong induction that for any positive integer $n$, the second player can always win the game when playing with two piles, each containing $n$ cards.

\textbf{Base Case:} For $n = 1$, there are two piles, each with one card. The first player must choose one pile and remove the only card, leaving the second player with no cards in the second pile, and the second player wins.

\textbf{Inductive Hypothesis:} Assume that for all positive integers $k$, where $1 \leq k \leq n$, the second player can win the game when playing with two piles, each containing $k$ cards.

\textbf{Inductive Step:} Consider the case when there are two piles, each containing $n + 1$ cards. The first player must choose one pile, and without loss of generality, let's say they choose the first pile and remove $x$ cards, where $1 \leq x \leq n$. This leaves two subcases:

\begin{enumerate}
\item The first player removes exactly one card from the first pile ($x = 1$). In this case, the game proceeds to two piles, one containing $n$ cards and the other containing $n + 1$ cards. Since $1 \leq n \leq n + 1$, by our inductive hypothesis, the second player can win the game when playing with these two piles. The second player mirrors the first player's moves in the subgame, always keeping the two piles in a configuration where the second player can win. When the subgame ends with one pile containing all the cards, the second player can win the main game.

\item The first player removes more than one card from the first pile ($x > 1$). In this case, the game proceeds to two piles, one containing $n - (x - 1)$ cards and the other containing $n + 1$ cards. Since $1 \leq n - (x - 1) \leq n$, by our inductive hypothesis, the second player can win the game when playing with these two piles. The second player mirrors the first player's moves in the subgame, always keeping the two piles in a configuration where the second player can win. When the subgame ends with one pile containing all the cards, the second player can win the main game.

\end{enumerate}

In both subcases, the second player can ensure victory in the subgame, and since the second player can mirror the first player's moves to achieve this, the second player can win the main game as well.

By the principle of strong mathematical induction, we have shown that the second player can always win the game when playing with two piles, each containing $n + 1$ cards, for any positive integer $n$.

Therefore, the second player can always win the game regardless of the number of cards in the piles.
\end{solution}



\begin{question}
    For each of the following functions, say if it is one-to-one and / or onto? Prove or disprove each statement.
    \begin{enumerate}
	\item $f : \Z \to \Z$ with $f(n) = n^2 + 1$ for $n \in \Z$.
	\item $f : \Z \to \Z$ with $f(n) = n/2$ if $n$ is even, and $f(n) = 0$ if $n$ is odd.
	\item $f : \R \to \R$ with $f(x) = 1/x$ if $x \neq 0$, and $f(0) = 0$.
	\item $f: \N \to \N$ with $f(n) = 2^n$ if $n$ is even and $f(n) = n$ if $n$ is odd.
	\item $f : \mathcal{P}(\Z) \to \mathcal{P}(\Z)$ with $f(A) = A \cup \{ 0 \}$ for $A \in \mathcal{P}(\Z)$.
    \end{enumerate}
\end{question}
% Student: put your answer between the next two lines.
\begin{solution}
\begin{enumerate}
    \item Let $f : \Z \to \Z$ be defined as $f(n) = n^2 + 1$ for $n \in \Z$. 

    To determine if $f$ is one-to-one, consider two integers, $a$ and $b$, such that $f(a) = f(b)$. This leads to the equation $a^2 + 1 = b^2 + 1$. Subtracting $1$ from both sides results in $a^2 = b^2$, and taking the square root gives $|a| = |b|$. Since $|a| = |b|$, it implies that either $a = b$ or $a = -b$. Thus, $f$ is not one-to-one because there exist distinct integers $a$ and $b$ such that $f(a) = f(b)$.

    To determine if $f$ is onto, consider an arbitrary integer $y$ in the codomain $\Z$. We need to find an integer $x$ in the domain $\Z$ such that $f(x) = y$. We have $f(x) = x^2 + 1 = y$. Solving for $x$ yields $x = \pm \sqrt{y - 1}$. Since $\sqrt{y - 1}$ might not always be an integer, $f$ is not onto because not all integers in the codomain can be represented as $f(x)$.

    \item Let $f : \Z \to \Z$ be defined as $f(n) = \begin{cases}
            \frac{n}{2} & \text{if } n \text{ is even} \\
            0 & \text{if } n \text{ is odd}
        \end{cases}$.

    To determine if $f$ is one-to-one, consider two integers, $a$ and $b$, such that $f(a) = f(b)$. We have two cases:
    
    Case 1: $a$ and $b$ are both even. In this case, $f(a) = \frac{a}{2}$ and $f(b) = \frac{b}{2}$. Since $\frac{a}{2} = \frac{b}{2}$ implies $a = b$, $f$ is one-to-one in this case.
    
    Case 2: $a$ and $b$ are both odd. In this case, $f(a) = f(b) = 0$. Since $0 = 0$, $f$ is one-to-one in this case.
    
    Therefore, $f$ is one-to-one in both cases, and hence, it is one-to-one.

    To determine if $f$ is onto, consider an arbitrary integer $y$ in the codomain $\Z$. We need to find an integer $x$ in the domain $\Z$ such that $f(x) = y$. If $y = 0$, then we can choose any odd integer $x$ to satisfy $f(x) = 0$. If $y$ is even, we can choose $x = 2y$ to satisfy $f(x) = \frac{x}{2} = y$. Therefore, $f$ is onto.

    \item Let $f : \R \to \R$ be defined as $f(x) = \begin{cases}
            \frac{1}{x} & \text{if } x \neq 0 \\
            0 & \text{if } x = 0
        \end{cases}$.

    To determine if $f$ is one-to-one, consider two real numbers, $a$ and $b$, such that $f(a) = f(b)$. We have two cases:

    Case 1: $a$ and $b$ are both non-zero real numbers. In this case, $f(a) = \frac{1}{a}$ and $f(b) = \frac{1}{b}$. Since $\frac{1}{a} = \frac{1}{b}$ implies $a = b$, $f$ is one-to-one in this case.

    Case 2: Either $a$ or $b$ (or both) is $0$. Without loss of generality, assume $a = 0$. In this case, $f(a) = 0$ and $f(b) = \frac{1}{b}$. Since $0 \neq \frac{1}{b}$ for any non-zero real number $b$, it is impossible for $f(a) = f(b)$ when $a = 0$. Therefore, $f$ is still one-to-one in this case.

    Since $f$ is one-to-one in both cases, it is one-to-one overall.

    To determine if $f$ is onto, consider an arbitrary real number $y$ in the codomain $\R$. We need to find a real number $x$ in the domain $\R$ such that $f(x) = y$. If $y = 0$, then we can choose $x = 0$ to satisfy $f(x) = 0$. If $y \neq 0$, we can choose $x = \frac{1}{y}$ to satisfy $f(x) = \frac{1}{x} = y$. Therefore, $f$ is onto.

    \item Let $f : \N \to \N$ be defined as $f(n) = \begin{cases}
            2^n & \text{if } n \text{ is even} \\
            n & \text{if } n \text{ is odd}
        \end{cases}$.

    To determine if $f$ is one-to-one, consider two natural numbers, $a$ and $b$, such that $f(a) = f(b)$. We have two cases:

    Case 1: Both $a$ and $b$ are even. In this case, $f(a) = 2^a$ and $f(b) = 2^b$. Since $2^a = 2^b$ implies $a = b$, $f$ is one-to-one in this case.

    Case 2: Both $a$ and $b$ are odd. In this case, $f(a) = a$ and $f(b) = b$. Since $a = b$, $f$ is one-to-one in this case.

    Therefore, $f$ is one-to-one in both cases, and hence, it is one-to-one overall.

    To determine if $f$ is onto, consider an arbitrary natural number $y$ in the codomain $\N$. We need to find a natural number $x$ in the domain $\N$ such that $f(x) = y$. If $y$ is even, we can choose $x = \log_2(y)$ (which is a natural number because $y$ is a power of $2$) to satisfy $f(x) = 2^x = y$. If $y$ is odd, we can choose $x = y$ to satisfy $f(x) = x = y$. Therefore, $f$ is onto.

    \item Let $f : \mathcal{P}(\Z) \to \mathcal{P}(\Z)$ be defined as $f(A) = A \cup \{ 0 \}$ for $A \in \mathcal{P}(\Z)$.

    To determine if $f$ is one-to-one, consider two subsets $A$ and $B$ in the power set $\mathcal{P}(\Z)$ such that $f(A) = f(B)$. This implies $A \cup \{ 0 \} = B \cup \{ 0 \}$. Removing $\{ 0 \}$ from both sides gives $A = B$. Thus, $f$ is one-to-one because distinct sets $A$ and $B$ lead to distinct values of $f(A)$ and $f(B)$.

    To determine if $f$ is onto, consider an arbitrary subset $Y$ in the codomain $\mathcal{P}(\Z)$. We need to find a subset $X$ in the domain $\mathcal{P}(\Z)$ such that $f(X) = Y$. Let $X = Y \setminus \{ 0 \}$, where $\setminus$ denotes set difference. Since $X$ is the result of removing $\{ 0 \}$ from $Y$, it follows that $f(X) = X \cup \{ 0 \} = Y \cup \{ 0 \} = Y$. Therefore, $f$ is onto.

\end{enumerate}
\end{solution}





\begin{question}
    \begin{enumerate}
        \item Let $A = \{1,2,3,4\}$ and $B = \{5,6,7\}$.
        Let $f$ be the relation $ \left\{(1,5),(2,5),(3,6),(?,?)\right\} $ where the question marks are to be filled in by you. 
        Give an example of $(?,?) \in A \times B$ so that: (Remember to explain your reasoning.)
            \begin{enumerate}
                \item The relation $f$ is not a function.
                \item The relation is a function from $A$ to $B$ but not onto $B$.
                \item The relation is a function from $A$ to $B$ and is onto $B$.
            \end{enumerate}

        \item     Let $A$ be an $n$-element set and let $i, j, k \in \mathbb{N}$ with $i+j+k = n$.
        How many functions $f \colon A \to \{0,1,2\}$ are there for which all three of the below are satisfied:
            \begin{itemize}
                \item $\left|\left\{ a \in A : f(a) = 0 \right\} \right| = i$
                \item $\left|\left\{ a \in A : f(a) = 1 \right\} \right| = j$
                \item $\left|\left\{ a \in A : f(a) = 2 \right\} \right| = k$
            \end{itemize}
    \end{enumerate}
\end{question}
% Student: put your answer between the next two lines.
\begin{solution}
\begin{enumerate}
    \item Let $A = \{1, 2, 3, 4\}$ and $B = \{5, 6, 7\}$. The relation $f$ is defined as $f = \{(1, 5), (2, 5), (3, 6), (?, ?)\}$, where we need to fill in the question marks.

    For part (a), we need to find values for $(?, ?)$ so that the relation $f$ is not a function. A relation is a function if each element in the domain ($A$) maps to a unique element in the codomain ($B$). To make $f$ not a function, we can assign the same value in $B$ to more than one element in $A$. For example, we can set $(4, 5)$, which means both 4 and 1 in $A$ map to 5 in $B$. This violates the definition of a function.

    For part (b), we need to find values for $(?, ?)$ so that the relation $f$ is a function from $A$ to $B$ but not onto $B$. This means all elements in $A$ must have a mapping in $B$, but $B$ should have at least one element without a preimage. We can achieve this by setting $(4, 7)$, which means all elements in $A$ have a mapping in $B, but 7 in $B$ has no preimage in $A$.

    For part (c), we need to find values for $(?, ?)$ so that the relation $f$ is a function from $A$ to $B$ and is onto $B$. This means every element in $B$ should have a preimage in $A$. We can achieve this by setting $(4, 7)$, which ensures that all elements in $B$ have preimages in $A.

    \item Let $A$ be an $n$-element set, and let $i, j, k \in \mathbb{N}$ with $i + j + k = n$. We want to count how many functions $f \colon A \to \{0, 1, 2\}$ satisfy the given conditions:

    \begin{itemize}
        \item $\left|\left\{ a \in A : f(a) = 0 \right\} \right| = i$
        \item $\left|\left\{ a \in A : f(a) = 1 \right\} \right| = j$
        \item $\left|\left\{ a \in A : f(a) = 2 \right\} \right| = k$
    \end{itemize}

    To count such functions, consider each element in $A$. We have three choices for each element: $0$, $1$, or $2$. Since there are $n$ elements in $A$, the total number of such functions is $3^n$ because each element can be assigned one of the three values independently. 

\end{enumerate}
\end{solution}


\begin{question}
    \begin{enumerate}
	\item There are five points inside an equilateral triangle of side length 2. 
	Show that at least two of the points are within 1 unit distance from each other.
	\item Let $A$ be a set of 10 distinct integers between 1 and 100 (both inclusive). 
	Show that there are two nonempty and disjoint subsets of $A$ such that the sum of all its elements are the same.
    \end{enumerate}
\end{question}
% Student: put your answer between the next two lines.
\begin{solution}
\begin{enumerate}
    \item Consider an equilateral triangle with side length 2. We place five points inside the triangle. Let's assume that no two points are within 1 unit of each other. This means that the minimum distance between any two points is at least 1 unit.

    Each point can be the center of a circle with a radius of 1 unit (a circle with radius 1 unit encompasses all points within 1 unit of that center). The circles around each of the five points do not overlap since the minimum distance between the points is 1 unit.

    Now, consider the areas covered by these circles. Since the circles don't overlap, their total area must be less than or equal to the area of the entire equilateral triangle.

    The area of the equilateral triangle with side length 2 is $\frac{\sqrt{3}}{4} \cdot 2^2 = 2\sqrt{3}$. 

    The area of each circle is $\pi \cdot 1^2 = \pi$, and since there are five circles, their total area is $5\pi$.

    To ensure that the circles do not overlap, their total area must be less than or equal to the area of the triangle:

    \[5\pi \leq 2\sqrt{3}.\]

    However, $\pi$ is approximately 3.1416, and $5\pi$ is larger than $2\sqrt{3}$, which is approximately 3.4641. This inequality contradicts the assumption that no two points are within 1 unit of each other, as the total area covered by the circles is greater than the area of the triangle. Therefore, at least two points must be within 1 unit of each other.

    \item Let $A$ be a set of 10 distinct integers between 1 and 100, both inclusive. We want to show that there are two nonempty and disjoint subsets of $A$ such that the sum of all their elements is the same.

    Consider the sums of all nonempty subsets of $A$. There are $2^{10}$ subsets of $A$ in total, including the empty set and $A$ itself. Since there are $2^{10}$ different subsets and only 100 possible sums of those subsets (ranging from 1 to 1000), there must be at least two distinct subsets that have the same sum (pigeonhole principle).

    Let these two subsets be $B$ and $C$, with $B \neq C$. If they are disjoint, we are done, as they are nonempty and have the same sum. If they are not disjoint, consider the set $B \cap C$.

    Since $B$ and $C$ are distinct, there is at least one element $x$ that is in $B$ but not in $C$, or vice versa (without loss of generality, assume $x \in B$ and $x \notin C$). Therefore, $x$ is an element of the set $B \cap C$.

    Now, consider the set $B' = B \setminus \{x\}$ and $C' = C \cup \{x\}$. Both $B'$ and $C'$ are subsets of $A$, they are distinct (as they differ by the element $x$), and their sums are the same as the sums of $B$ and $C$, respectively.

    So, we've found two nonempty and disjoint subsets of $A$ (namely, $B'$ and $C'$) with the same sum. This completes the proof.
\end{enumerate}
\end{solution}



\begin{question}
    Let $A = \{ x \in \Z ~:~ 3|x \}$.  Show that $A$ and $\mathbb{N}$ has the same cardinality.  {\it Hint:} Define $f: A \rightarrow \N$ such that
\[ f(x) =\left\{\begin{array}{c c l} \frac{2}{3}x & & \text{if } x \geq 0 \\ ~ \\ -\frac{2}{3}x - 1 &  & \text{if } x < 0 \end{array} \right. .\]
\end{question}
% Student: put your answer between the next two lines.
\begin{solution}
We want to show that the sets $A$ and $\mathbb{N}$ have the same cardinality by defining a bijection $f: A \to \mathbb{N}$.

We'll define the function $f$ as provided in the hint:
\[ f(x) =\left\{
\begin{array}{lll}
\frac{2}{3}x & \text{if } x \geq 0, \\
-\frac{2}{3}x - 1 & \text{if } x < 0.
\end{array}
\right.
\]

To prove that $f$ is a bijection, we need to show that it is both injective (one-to-one) and surjective (onto).

\begin{enumerate}
\item \textbf{Injectivity:} We'll prove that $f$ is injective by considering cases. Let $x_1, x_2 \in A$ such that $f(x_1) = f(x_2)$.

\begin{itemize}
\item If $x_1 \geq 0$ and $x_2 \geq 0$, then $\frac{2}{3}x_1 = \frac{2}{3}x_2$, which implies $x_1 = x_2$.

\item If $x_1 < 0$ and $x_2 < 0$, then $-\frac{2}{3}x_1 - 1 = -\frac{2}{3}x_2 - 1$, which implies $-\frac{2}{3}x_1 = -\frac{2}{3}x_2$. Canceling out the common factor, we get $x_1 = x_2$.

\item If $x_1 < 0$ and $x_2 \geq 0$, then $-\frac{2}{3}x_1 - 1 = \frac{2}{3}x_2$, which simplifies to $-\frac{2}{3}x_1 = \frac{2}{3}x_2 + 1$. This equation has no solutions because the left side is negative, and the right side is non-negative.

\item If $x_1 \geq 0$ and $x_2 < 0$, then $\frac{2}{3}x_1 = -\frac{2}{3}x_2 - 1$, which simplifies to $\frac{2}{3}x_1 + \frac{2}{3}x_2 = -1$. This equation also has no solutions because the left side is non-negative, and the right side is negative.

\end{itemize}

In all cases, we've shown that if $f(x_1) = f(x_2)$, then $x_1 = x_2$, proving that $f$ is injective.

\item \textbf{Surjectivity:} We need to show that for every $n \in \mathbb{N}$, there exists an $x \in A$ such that $f(x) = n$. We'll consider cases again.

\begin{itemize}
\item For $n \in \mathbb{N}$, if $n$ is even ($n = 2k$ for some $k \in \mathbb{N}$), we can find $x \in A$ such that $f(x) = n$ by letting $x = \frac{3}{2}k$ (which is non-negative). In this case, $f(x) = \frac{2}{3}x = \frac{2}{3} \cdot \frac{3}{2}k = k = n$.

\item For $n \in \mathbb{N}$, if $n$ is odd ($n = 2k + 1$ for some $k \in \mathbb{N}$), we can find $x \in A$ such that $f(x) = n$ by letting $x = -\frac{3}{2}k - 1$ (which is negative). In this case, $f(x) = -\frac{2}{3}x - 1 = -\frac{2}{3} \cdot \left(-\frac{3}{2}k - 1\right) - 1 = k + 1 - 1 = k = n$.

\end{itemize}

In both cases, we've shown that for any $n \in \mathbb{N}$, we can find an $x \in A$ such that $f(x) = n$, proving that $f$ is surjective.

Since $f$ is both injective and surjective, it is a bijection, which means that $A$ and $\mathbb{N}$ have the same cardinality.
\end{enumerate}
\end{solution}






\end{document}
