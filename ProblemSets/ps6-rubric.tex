\documentclass{article}
% The LaTeX macro language is complicated, so we have inserted
% lots of documenting comments into the file.  Comments start
% with `%' and continue to the end of the line.  In Overleaf's
% window, they are colored green
%
% Comments prefixed with `Student:' are relevant to students.
% Skip anything else you don't understand, or ask me.
%
% set font encoding for PDFLaTeX or XeLaTeX
\usepackage{ifxetex}
\ifxetex
  \usepackage{fontspec}
\else
  \usepackage[T1]{fontenc}
  \usepackage[utf8]{inputenc}
  \usepackage{lmodern}
\fi

% Student: These lines describe some document metadata.
\title{Problem Set 5}
\author{%
% Student: change the next line to your name!
    Name
\\  MATH-UA 120 Discrete Mathematics
}
\date{due October 27, 2023}


\usepackage[headings=runin-fixed-nr]{exsheets}
% These make enumerates within questions start at the second ("(a)") level, rather than the first ("1.") level.
\makeatletter
    \newcommand{\stepenumdepth}{\advance\@enumdepth\@ne}
\makeatother
\SetupExSheets{
    question/pre-body-hook=\stepenumdepth,
    solution/pre-body-hook=\stepenumdepth,
}
\DeclareInstance{exsheets-heading}{runin-nn-np}{default}{
    runin = true,
    title-post-code = .\space,
    join = {
        main[r,vc]title[l,vc](0pt,0pt);
    }
}
\newif\ifshowsolutions
% Student: replace `false' with `true' to typeset your solutions.
% Otherwise they are ignored!
\showsolutionstrue
\ifshowsolutions
    \SetupExSheets{
        question/pre-hook=\itshape,
        solution/headings=runin-nn-np,
        solution/print=true,
        solution/name=Answer
    }%
    \makeatletter%
    \pretocmd{\@title}{Answers to }%
    \makeatother%
\else
    \SetupExSheets{solution/print=false}
\fi

% Bug workaround: http://tex.stackexchange.com/a/146536/1402
%\newenvironment{exercise}{}{}
\RenewQuSolPair{question}{solution}
%\let\answer\solution
%\let\endanswer\endsolution
\usepackage{manfnt}
\newcommand{\danger}{\marginpar[\hfill\dbend]{\dbend\hfill}}

% We are creating a command for some common commands.
\newcommand{\Z}{\mathbb{Z}}

% This package is for specifying graphics.  It's amazing.
% Manual at http://texdoc.net/texmf-dist/doc/generic/pgf/pgfmanual.pdf
\usepackage{tikz}

\usepackage{amsmath, amsthm, amssymb}
\usepackage{amsfonts}
\usepackage{siunitx}
\DeclareSIUnit\pound{lb}
\usepackage{hyperref}
\newtheorem*{theorem}{Theorem}
\theoremstyle{definition}
\newtheorem*{definition}{Definition}
% This is the beginning of the part of the file that describes
% the text of the document.
% That's why it says `\begin{document}' below. :-)
\begin{document}
\maketitle



These are to be written up in \LaTeX{} and turned in to Gradescope.\\



\ifshowsolutions
    \SetupExSheets{solution/print=true}
\else
    \danger
 \underline{ \LaTeX{}  Instructions:}  You can view the source (\texttt{.tex}) file to get some more examples of \LaTeX{} code.  I have commented the source file in places where new \LaTeX{} constructions are used.
  
  Remember to change \verb|\showsolutionsfalse| to \verb|\showsolutionstrue|
    in the document's preamble 
    (between \verb|\documentclass{article}| and \verb|\begin{document}|)
\fi

\section*{Assigned Problems}


% ADS by Doerr & Levasseur
% Sec 3.7 Q. 7
\begin{question}
    Let $n\geq 1$. Suppose we want a bit pattern of length $n$ (i.e. a sequence of 1's and 0's of length $n$ such as $011001010...$). Use induction to prove there are $2^{n-1}$ bit patterns with an even number of 1's. 
\end{question}
% Student: put your answer between the next two lines.
\begin{solution}
        Note there are $2^n$ bit patterns. 
	\begin{description}
	\item[Base Cases: ] Consider $n=1$. With length one, only the sequence of one 0 has an even number of 1's. There is $1=2^{1-1} = 2^0$ bit pattern with an even number of 1's.
	
	\item[Inductive Hypothesis: ] Suppose for a bit pattern of length $n=k$ for some $k\geq 1$, there are $2^{k-1}$ bit patterns with an even number of 1's.
	
	\item[Inductive Step: ] Consider $n=k+1$. Let $E$ be the set of bit patterns with an even number of 1's in the first $k$ positions and the ($k+1$)-st position will be a 0 , and $O$ be the set of bit patterns with an odd number of 1's in the first $k$ positions and the ($k+1$)-st position will be a 1. Note $E\cap O=\emptyset$. The set of all $k+1$-length bit patterns with an even number of 1's is $E\cup O$. Note that $|E| = 2^{k-1}$ because, by the inductive hypothesis, there are $2^{k-1}$ possible choices for the first $k$ positions and only 1 choice for the ($k+1$)-st position, and $|O|=2^{k} -2^{k-1}$ because out of all possible bit patterns for the first $k$ positions, we removed all possible bit patterns with an even number of 1's in the first $k$ position, and there's only one choice for the ($k+1$)-st position. Then $|E\cup O| = |E| + |O| = 2^{k-1}+2^k-2^{k-1} = 2^k$. 
	\end{description}
	Therefore, by the principle of mathematical induction, the statement is true for $n\geq 1$.
	
{\color{red} Rubric:
\begin{itemize}
\item Follow RVF rubric with 1P for \LaTeX
\end{itemize}}
\end{solution}








\end{document}
